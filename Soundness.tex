\subsection{Soundness of the rules above}

    To prove soundness, we first define the following: 

    \begin{definition}{Implied Write Order(\emph{iwo})}
        \label{ImpliedW}
        Binary relation between any two \emph{distinct} writes of symmetric threads, derived through the following two sequential conposition:  
        \begin{align*}
            w_i^j;po;rf^{-1};w_k^j \\
            w_i^j;rf;po;w_k^j 
        \end{align*}
    \end{definition}

%---------------------------------------------------------------------------------------------------------------------------------------

    \begin{property}
        \label{Wabv}
        $w_i^j;po;rf^{-1};w_k^j$ corresponds to an implied write order between writes above the read((po before read)) and $T_i$'s read being satisfied. 
    \end{property}
    
    \begin{proof}
        Expanding the first sequential composition exposes the following relations involved. 
        \begin{align*}
            \reln{w_i^j}{po}{r_i} \\ 
            \reln{r_i}{rf^{-1}}{w_k^j}
        \end{align*}
        By Def \ref{ProgO}, $w_i^j$ is above its read $r_i$. Because $T_i$ and $T_k$ are symmetric threads, we can infer $\reln{w_k^j}{po}{r_k}$, thus verifying that $w_k^j$ is also above its read. 
        By Def \ref{ReadF}, we can infer $\reln{w_k^j}{rf}{r_i}$ implying $T_i$'s read being satisfied.   
    \end{proof}

%-------------------------------------------------------------------------------------------------------------------------------------

    \begin{property}
        \label{Wbel}
        $w_i^j;rf;po;w_k^j$ corresponds to an implied write order between writes below the read (po after read) and $T_k$'s read being satisfied.
    \end{property}

    \begin{proof}
        Expanding the first sequential composition exposes the following relations involved. 
        \begin{align*}
            \reln{w_i^j}{rf}{r_k} \\ 
            \reln{r_k}{po}{w_k^j}
        \end{align*}
        By Def\ref{ProgO}, $w_k^j$ is below its read $r_k$. Because $T_i$ and $T_k$ are symmetric threads, we can infer $\reln{r_i}{po}{w_i^j}$, thus verifying that $w_i^j$ is also below its read.
        By Def \ref{ReadF}, we can infer $T_k$'s read being satisfied. 
    \end{proof}

%-------------------------------------------------------------------------------------------------------------------------------------

    \begin{property}{Simplified irreflexivity rule}
        \label{prop1}
        The irreflexivity constraint rule is equivalent to the following irreflexivity condition 
        \begin{align*}
            smo;iwo
        \end{align*}
    \end{property}

    \begin{proof}
        Expanding for implied write order as per the definition, gives us the following two sequential compositions. 
        \begin{align*}
            smo;w_i^j;po;rf^{-1};w_k^j \\
            smo;w_i^j;rf;po;w_k^j 
        \end{align*}
        If the above relations must be irreflexive, then so should the following: 
        \begin{align*}
            w_k^j;smo;w_i^j;po;rf^{-1} \\
            w_k^j;smo;w_i^j;rf;po     
        \end{align*}
        By Def \ref{SymMemO}, the above can be simplified to
        \begin{align*}
            smo;po;rf^{-1} \\
            smo;rf;po 
        \end{align*}
        Thus, proving our property. 
    \end{proof}

%-----------------------------------------------------------------------------------------------------------------------------------

    \begin{property}
        \label{prop2}
        No write order is implied when a read reads from its own thread's write
    \end{property}

    \begin{proof}
        If the read is from its own thread's write, then we can infer that $i=k$ in both the sequential compositions. Hence  
        \begin{align*}
            w_i^j;rf;po;w_i^j \\
            w_i^j;po;rf^{-1};w_i^j
        \end{align*} 
        which gives us $\reln{w_i^j}{iwo}{w_i^j}$.
        Since implied write orders are only between distinct writes, the property is proven.  
    \end{proof}

    \critic{red}{Perhaps define implied write order as being irreflexive}
        
%-------------------------------------------------------------------------------------------------------------------------------------

    \begin{property}
        \label{prop3}
        Implied write orders between two symmetric threads are reversed when they are swapped.
    \end{property}
        
    \begin{proof}
        Considering first sequential composition, i.e. $w_i^j;rf;po;w_k^j$, expanding gives us the following binary relations involved:
        \begin{align*}
            \reln{w_i^j}{rf}{r_k} \\
            \reln{r_k}{po}{w_k^j}
        \end{align*}
        Swapping thread identities involves swapping the indices $i$ and $k$ for each event, thus giving us 
        \begin{align*}
            \reln{w_k^j}{rf}{r_i} \\
            \reln{r_i}{po}{w_i^j}
        \end{align*}
        Through sequential composition of the above, we get $w_k^j;rf;po;w_i^j$, which by definition is $\reln{w_k^j}{iwo}{w_i^j}$.

        For the second sequential composition, i.e. $w_i^j;po;rf^{-1};w_k^j$, expanding gives us the following binary relations involved:
        \begin{align*}
            \reln{w_i^j}{po}{r_i} \\
            \reln{r_i}{rf^{-1}}{w_k^j}
        \end{align*}
        Swapping thread identities $T_i$ and $T_k$ gives us the following relations 
        \begin{align*}
            \reln{w_k^j}{po}{r_k} \\
            \reln{r_k}{rf^{-1}}{w_i^j}
        \end{align*}
        Whose sequential compostiion gives us $w_k^j;po;rf^{-1};w_i^j$, which by defintion is $\reln{w_k^j}{iwo}{w_i^j}$.
    \end{proof}
        
%-------------------------------------------------------------------------------------------------------------------------------------

    \begin{property}
        \label{prop4}
        There are at most two implied write orders between writes of two threads, with one between writes above the read event and one below. 
    \end{property}
        
    \begin{proof}
        Consider two threads $T_i$ and $T_k$. Suppose we have one implied write order between one of their writes, i.e. 
        \begin{align*}
            \reln{w_i^j}{iwo}{w_k^j}.    
        \end{align*}

        If the above is derived by $w_i^j;po;rf^{-1};w_k^j$ then from Prop \ref{Wabv}, we can infer the two writes are above the read and $T_i$' read being satisfied.  

        Now suppose we have an implied write order between another set of writes, i.e.
        \begin{align*}
            \reln{w_i^l}{iwo}{w_k^l}.    
        \end{align*}
        If the above is derived by $w_i^l;po;rf^{-1};w_k^l$, then by Prop \ref{Wabv} and Def \ref{ReadF}, it violates $\stck{_{rf}}$ functionality. Hence it can only be derived by $w_i^l;rf;po;w_k^l$. From Prop \ref{Wbel}, we can infer that the two writes are below the read and $T_k$'s read being satisfied.  
        
        Suppose a third impleid write order exists of the form 
        \begin{align*}
            \reln{w_i^n}{iwo}{w_k^n}
        \end{align*}
        If the above is derived by $w_i^n;po;rf^{-1};w_k^n$, then by Prop \ref{Wabv} and Def \ref{ReadF}, it violates $\stck{_{rf}}$ functionality. If the above is derived by $w_i^n;rf;po;w_k^n$, then by Prop \ref{Wbel} and Def \ref{ReadF}, it violates $\stck{_{rf}}$ functionality.
        
        Hence, we cannot have any more implied write orders between $T_i$ and $T_j$, thus verifying our property. 
    \end{proof}
        
%--------------------------------------------------------------------------------------------------------------------------------------

    \begin{property}
        \label{prop5}
        Implied write orders between two threads either all respect irreflexivity condition by Prop \ref{prop1} or they all do not.  
    \end{property}

    \begin{proof}
        
        Consider our two threads $T_i$ and $T_k$. 

        For cases where implied write orders are established between wrties of $T_i$ and $T_j$, let us first consider one between writes above the read which respect Prop \ref{prop1}; 
        \begin{align*}
            \reln{w_i^j}{iwo}{w_k^j} \ \wedge \ \reln{w_i^j}{smo}{w_k^j}
        \end{align*}
        By Prop \ref{Wabv}, the implied write order satisfies $T_i$'s read by a reads-from relation with $w_k^j$. 

        The other set of implied write order, by Prop \ref{prop4} can only be between writes below the read. Suppose we have such an order but not respecting Prop \ref{prop1}:
        \begin{align*}
            \reln{w_k^l}{iwo}{w_i^l} \ \wedge \ \reln{w_i^l}{smo}{w_k^l}
        \end{align*}
        By Prop \ref{Wbel}, this implies another $\stck{_{rf}}$ relation with $T_i$'s read. By Def \ref{ReadF} this violates the functional property of $\stck{_{rf}}$. Hence we can only have an implied write order respecting Prop \ref{prop1}.
        \begin{align*}
            \reln{w_i^l}{iwo}{w_k^l} \ \wedge \ \reln{w_i^l}{smo}{w_k^l}
        \end{align*}

        By symmetry, if the implied write order between writes above the read did not respect Prop \ref{prop1}, then so would the writes below the read.  
    \end{proof}

%-------------------------------------------------------------------------------------------------------------------------------------- 
        
    \begin{property}
        \label{prop6}
        Implied write orders are acyclic
    \end{property}

    \begin{proof}
        %By Contradiction 
        Suppose a cycle exists. Then without loss of generality, we can consider the cycle composed of 3 writes.
        \begin{align*}
            \reln{w_i^j}{iwo}{w_k^j} \ \wedge \reln{w_k^j}{iwo}{w_l^j} \ \wedge \reln{w_l^j}{iwo}{w_i^j}.  
        \end{align*}
        
        By Def \ref{ImpliedW} and Prop \ref{Wabv}, \ref{Wbel}, we can have just two cases; the set of writes involved in cycle are either above the read event or below.

        If they are above the read, then we have the following relations that result in the cycle
        \begin{align*}
            \reln{w_i^j}{po}{r_i} \ \wedge \ \reln{r_i}{rf^{-1}}{w_k^j} \\
            \reln{w_k^j}{po}{r_k} \ \wedge \ \reln{r_k}{rf^{-1}}{w_l^j} \\
            \reln{w_l^j}{po}{r_l} \ \wedge \ \reln{r_l}{rf^{-1}}{w_i^j}. 
        \end{align*}
        The above relations violate $po \cup rf^{-1} \ \text{acyclic}$ rule, thus violating coherence.
        
        If the writes are below the read, then we have the following relations that result in the above cycle.
        \begin{align*}
            \reln{w_i^j}{po}{r_i} \ \wedge \ \reln{r_i}{rf}{w_k^j} \\
            \reln{w_k^j}{po}{r_k} \ \wedge \ \reln{r_k}{rf}{w_l^j} \\
            \reln{w_l^j}{po}{r_l} \ \wedge \ \reln{r_l}{rf}{w_i^j} 
        \end{align*}
        The above relations form a cycle thus violating $po \cup rf \ \text{acyclic}$ rule, thus violating coherence.

        Because both cases violate coherence, we conclude that $\stck{_{iwo}}$ is acylic. 
    \end{proof}

%---------------------------------------------------------------------------------------------------------------------------------------

    \begin{property}
        \label{prop7}
        Writes $w_i$ above read in each thread can only have one relation of the form $\reln{w_i}{iwo}{w}$ but can have several of the form $\reln{w}{iwo}{w_i}$. 
    \end{property}

    \begin{proof}
        Suppose a write $w_i^j$ above a read has a relation with $w_k^j$ such that $\reln{w_i^j}{iwo}{w_k^j}$. By Prop \ref{Wabv}, we can infer $T_i$'s read has been satisfied. By Def \ref{ReadF} and our assumption of one read per thread, we cannot have any more implied write order relations of the form $\reln{w_i^j}{iwo}{w^j}$. 

        On the other hand, there can be many relations of the form $\reln{w_k^j}{iwo}{w_i^j}$ as $k$ can be identity of any thread, whose read gets satisfied by the argument above.
    \end{proof}

%---------------------------------------------------------------------------------------------------------------------------------------
   
    \begin{property}
        \label{prop8}
        Writes $w_i$ below read in each thread can only have one relation of the form $\reln{w}{iwo}{w_i}$ but can have several of the form $\reln{w_i}{iwo}{w}$. 
    \end{property}

    \begin{proof}
        Suppose a write $w_i^j$ below the read has a relation with $w_k^j$ such that $\reln{w_k^j}{iwo}{w_i^j}$. By Prop \ref{Wbel}, we can infer that $T_i$'s read has been satisfied by write $w_k^j$. By Def \ref{ReadF} and our assumption of one read per thread, we cannot have any more implied write order relations of the form $\reln{w^j}{iwo}{w_i^j}$. 

        On the other hand, there can be many relations of the form $\reln{w_i^j}{iwo}{w_k^j}$ as $k$ can be identity of any thread, whose read gets satisfied by the argument above.  
    \end{proof}

%----------------------------------------------------------------------------------------------------------------------------------------
%----------------------------------------------------------------------------------------------------------------------------------------

    \subsubsection{Soundness}

        %Point 1
        To prove that our rules are sound, we need to show that for every execution graph that does not respect the irreflexivity constraint, there is a symmetric execution that does respect and hence covered by our rules. 
        
        %Point 2
        By Prop \ref{prop1}, we can infer that an execution not respectiing our irreflexivity constraint will have implied write orders which are "incorrect". 

        \critic{red}{Perhaps put Incorrect implied write orders as a definition??}
        
        %Point 3
        By Prop \ref{prop3}, we can swap thread identities to reverse implied write orders, thus "fixing" implied write orders, which then can make the resultant execution respect our rules. If we can show this process of "fixing" implied write orders is terminating, our soundness proof is complete. 

        %Point 4
        We also fix the order in which we do such "fixing". We first address all implied write orders above the read and then go below. 
        If any new implied write orders is introduced in the process, then we fix them as per the above order, i.e. if the new relation is above the read, we fix that next, contrast to any new order introduced below the read. 
        
        %Point 
        However, for each new implied write order introduced, we need to show that it wasn't "fixed" before, hence giving us guarantee that "fixing" is a terminating process. 
        
        %Point 5
        However, before all this, we first need to ensure that implied write orders do not form a cycle, as "fixing" them may not terminate.  By Prop \ref{prop6}, we have that $\stck{_{iwo}}$ is acyclic, hence they do not form a cycle.  
        
        %Point 6
        Given this, we show that for one given set of equal writes, swapping thread identities will always result in the implied write orders respecting Prop \ref{prop1} (Lemma 1). 
        We then show that this holds in general given multiple sets of equal writes (Lemma 2).  

%------------------------------------------------------------------------------------------------------------------------------------------

    %Lemma 1: While fixing one set of equal writes, once an implied write order is fixed, it remains fixed. 

        \begin{lemma}
            For one given set of equal writes, every implied write order that is "incorrect", can be fixed and will remain fixed.  
            (A more formal statement required perhaps?)
        \end{lemma}
            
        \begin{proof}
            %Proof by contradiction 
            Suppose we have one implied write order which is correct / fixed between writes $w_i^j$ and $w_k^j$.
            \begin{align*}
                \reln{w_i^j}{iwo}{w_k^j}
            \end{align*}
            
            We divide our proof into two cases, one for writes above the read and one for writes below. 

            \paragraph{Case1 : The writes are above the read} 
                
                \begin{itemize}
                    \item Part 1: There exists an implied write order between $w_i^j$ and some $w_m^j$
                    
                        Because Prop \ref{prop7} and $\reln{w_i^j}{iwo}{w_k^j}$, other implied write orders with $w_i^j$ can be of the form:
                        \begin{align*}
                            \reln{w_m^j}{iwo}{w_i^j}
                        \end{align*}
                        If this implied order is wrong, swapping thread identities will  give us the following relations 
                        \begin{align*}
                            \reln{w_i^j}{iwo}{w_m^j} \ \wedge \ \reln{w_m^j}{iwo}{w_k^j}  
                        \end{align*}  
                        Here, there is no implied write order relation between $w_i^j$ and $w_k^j$, hence we can consider them as remaining fixed. 
                        If the implied order between $w_m^j$ and $w_k^j$ is wrong, swapping it will result in the following relations
                        \begin{align*}
                            \reln{w_i^j}{iwo}{w_k^j} \ \wedge \ \reln{w_k^j}{iwo}{w_m^j}  
                        \end{align*}  
                        Here, the implied write order between $w_i^j$ and $w_k^j$ remains the same. Thus, for this case, we can conclude    that   the   relation remains "fixed".

                    \item Part 2: There exists an implied write order between $w_k^j$ and some $w_m^j$
                
                        By Prop \ref{prop7}, implied write orders with $w_k^j$ can be of the forms:
                        \begin{align*}
                            \reln{w_k^j}{iwo}{w_m^j} \\
                            \reln{w_m^j}{iwo}{w_k^j} 
                        \end{align*} 
                        The first form is symmetric to Case1, hence we only consider the second form. 

                        If this implied write order is wrong, swapping thread identities will give us the following relations. 
                        \begin{align*}
                            \reln{w_i^j}{iwo}{w_m^j} \ \wedge \ \reln{w_k^j}{iwo}{w_m^j}
                        \end{align*}
                        Here, there is no implied write order relation between $w_i^j$ and $w_k^j$, hence we can consider them as remaining fixed. 

                        If the implied write order between $w_m^j$ and $w_i^j$ is wrong, swapping their threads will give us 
                        \begin{align*}
                            \reln{w_m^j}{iwo}{w_i^j} \ \wedge \ \reln{w_k^j}{iwo}{w_i^j}  
                        \end{align*}
                        thus making our claim invalid. To show that this state is not possible, note firstly that from the initial  configuration,   we    can infer by Def \ref{SymMemO} and Prop \ref{prop1}:
                        \begin{align*}
                            \reln{w_i^j}{smo}{w_k^j} \ \wedge \ \reln{w_k^j}{smo}{w_m^j}
                        \end{align*} 
                        Because $smo$ is a total order w.r.t. one set of writes, we have by transitivity. 
                        \begin{align*}
                            \reln{w_i^j}{smo}{w_m^j}
                        \end{align*}
                        After swapping threads $T_k$ and $T_m$, we get $\reln{w_i^j}{iwo}{w_m^j}$ which respects the irreflexivity constraint   Prop  \ref {prop1}. Hence, this implied write order is not wrong. Thus we cannot have the case which results in $\reln{w_k^j}{iwo}{w_i^j}$.  
                        
                    \end{itemize}

            \paragraph{Case2: The writes are below the read} 

                \begin{itemize}
                    \item Part 1: There exists an implied write order between $w_i^j$ and some $w_m^j$
                    
                        By Prop \ref{prop8}, implied write order between $w_i^j$ and $w_m^j$ can be of two forms 
                        \begin{align*}
                            \reln{w_m^j}{iwo}{w_i^j} \\ 
                            \reln{w_i^j}{iwo}{w_m^j}
                        \end{align*}

                        Considering the first form, if it is wrong, swapping it will result in the following relations 
                        \begin{align*}
                            \reln{w_i^j}{iwo}{w_m^j} \ \wedge \ \reln{w_m^j}{iwo}{w_k^j}
                        \end{align*}
                        Here, there is no implied write order relation between $w_i^j$ and $w_k^j$, hence we can consider them as remaining fixed. 

                        If the implied write order between $w_m^j$ and $w_k^j$ is wrong, swapping their threads will give us
                        \begin{align*}
                            \reln{w_i^j}{iwo}{w_k^j} \ \wedge \ \reln{w_k^j}{iwo}{w_m^j}
                        \end{align*} 
                        Here, the implied write order between $w_i^j$ and $w_k^j$ remains the same, while we fixed the relations they had with $w_m^j$. Thus, for this case, we can conclude that the implied write order remains "fixed".

                        Considering the second form, if it is wrong, swapping it will result in the following relations 
                        \begin{align*}
                            \reln{w_m^j}{iwo}{w_i^j} \ \wedge \ \reln{w_m^j}{iwo}{w_k^j}
                        \end{align*}

                        If the implied write order between $w_m^j$ and $w_i^j$ is wrong, swapping their threads will give us 
                        \begin{align*}
                            \reln{w_k^j}{iwo}{w_m^j} \ \wedge \ \reln{w_k^j}{iwo}{w_i^j}  
                        \end{align*}
                        thus making our claim invalid. To show that this state is not possible, note firstly that from the initial  configuration, we can infer by Def \ref{SymMemO} and Prop \ref{prop1} (given relation between $w_i^j$ and $w_m^j$ is wrong):
                        \begin{align*}
                            \reln{w_m^j}{smo}{w_i^j} \ \wedge \ \reln{w_i^j}{smo}{w_k^j}
                        \end{align*}
                        Because $smo$ is a total order w.r.t. one set of writes, we have by transitivity. 
                        \begin{align*}
                            \reln{w_m^j}{smo}{w_k^j}
                        \end{align*}

                        After swapping threads $T_i$ and $T_m$, we get $\reln{w_m^j}{iwo}{w_k^j}$ which respects the irreflexivity constraint Prop \ref{prop1}. Hence, this implied write order is not wrong. Thus we cannot have the case which results in $\reln{w_k^j}{iwo}{w_i^j}$.

                    \item Part 2: There exists an implied write order between $w_k^j$ and some $w_m^j$
                    
                        By Prop \ref{prop8} and $\reln{w_i^j}{iwo}{w_k^j}$, relation between $w_k^j$ and $w_m^j$ can only be of one form 
                        \begin{align*}
                            \reln{w_k^j}{iwo}{w_m^j}
                        \end{align*}
                        This case is symmetric to the first form of Part 1, hence this case is already proved. 
                
                \end{itemize}
 
            Thus, for a given set of equal writes, once an implied write order is fixed, it remains fixed.
        \end{proof}

%------------------------------------------------------------------------------------------------------------------------------------------

    %Lemma 2: While fixing one set of implied write orders, any other implied write orders introduced between another set of writes has not been fixed before. 

        \begin{lemma}
            For a given set of equal writes whose implied write orders respect Prop \ref{prop1}, new implied write order introduced among them by fixing other sets, if wrong, can be fixed and will remain fixed. 
            (A more formal statement required? Discuss with Viktor)        
        \end{lemma}

        \begin{proof}{Main part}
            We go about the proof by considering a set of writes fixed and another set that needs fixing. 
            We then show that on fixing an implied write order, any new implied write order to the fixed set may either be right, or if wrong, was not something fixed before, and hence can be fixed by lemma 1. 
            Since we fixed the order in which we "fix" these orders, we have to only consider the following cases:
            \paragraph{1. Write orders above read fixed , write orders above still need fixing.}

                Let us assume that the fixed set \textbf{above the read is the set of writes $w^j$}. Let us consider two threads $T_i$ and $T_k$ whose writes \textbf{above the read $w_i^l$ and $w_k^l$} is going to be fixed. Without loss of generality, let us conisder the symmetric memory order between these two threads to be of the form $\reln{w_i}{smo}{w_k}$.  By Prop \ref{prop5}, we then have the configuration as:
                \begin{align*}
                    \reln{w_i^j}{smo}{w_k^j} \ \wedge \ \reln{w_k^l}{iwo}{w_i^l}
                \end{align*}

                We divide our analysis into two cases: 
                
                \begin{itemize}
                    \item There is an implied write order with $w_i^j$ and some $w^j$

                        By Prop \ref{prop7}, implied write orders between the two events can be of the form 
                        \begin{align*}
                            \reln{w^j}{iwo}{w_i^j} \\
                            \reln{w_i^j}{iwo}{w^j}
                        \end{align*}

                        For the first case, note that by Def \ref{SymMemO} and Prop \ref{prop1}, we have 
                        \begin{align*}
                            \reln{w*j}{smo}{w_k^j}
                        \end{align*}
                        By Prop \ref{prop3}, on swapping $T_i$ and $T_k$, we have the following relations 
                        \begin{align*}
                            \reln{w_i^l}{iwo}{w_k^l} \ \wedge \ \reln{w^j}{iwo}{w_k^j}
                        \end{align*}

                        The above relations do not violate Prop \ref{prop1}, hence we can stop here. 

                        For the second case, by Prop \ref{prop3}, on swapping $T_i$ and $T_k$, we have the following relations 
                        \begin{align*}
                            \reln{w_i^l}{iwo}{w_k^l} \ \wedge \ \reln{w_k^j}{iwo}{w^j}
                        \end{align*}
                        The new relation $\reln{w_k^j}{iwo}{w^j}$ may be wrong. 
                        To show that this relation was not fixed before, we consider cases where it could have been fixed. We then show that such a configuration cannot exist. For this, we consdier another event $x$, and the following cases:
                        
                        \begin{itemize}
                            \item Case where $\reln{w^j}{iwo}{x} \ \wedge \ \reln{x}{iwo}{w_k^j}$
                                %po union rf-1 cycle, violating coherence 
                            \item Case where $\reln{x}{iwo}{w^j} \ \wedge \ \reln{x}{iwo}{w_k^j}$
                                %By property 7, x is invalid
                            \item Case where $\reln{w^j}{smo}{x} \ \wedge \ \reln{x}{iwo}{w_k^j}$
                                %There could exist events w' and x' above the read program ordered with w^j and x resp, with an implied write order. This could have been fixed before, thus giving us a relation with w^j and w_k^j, which could have also been fixed. 
                                %But such a configuration cannot exist, because po union rf-1 is a cycle here, which violates coherence
                            \item Case where $\reln{x}{iwo}{w^j} \ \wedge \ \reln{x}{iwo}{w_k^j}$
                                %There could exist events w' and x' above the read program ordered with w^j and x resp, with an implied write order. This could have been fixed before, thus giving us a relation with w^j and w_k^j, which could have also been fixed.
                                %By Def \ref{SymMemO} and Prop \ref{prop1}, we have $\reln{x'}{iwo}{w'}$. By Prop \ref{prop7} and Prop \ref{Wabv}, either $x'$ or $x$ is invalid, hence such a configuration cannot exist. 
                        \end{itemize}


                    \item There is an implied write order with $w_k^j$ and some $w^j$
                    
                        By Prop \ref{prop7}, implied write orders between the two events can be of the form 
                        \begin{align*}
                            \reln{w^j}{iwo}{w_k^j} 
                        \end{align*}
                        
                        For the above case, by Prop \ref{prop3}, on swapping $T_i$ and $T_k$, we have the following relations 
                        \begin{align*}
                            \reln{w_i^l}{iwo}{w_k^l} \ \wedge \ \reln{w^j}{iwo}{w_i^j}
                        \end{align*}
                        The new relation $\reln{w^j}{iwo}{w_i^j}$ may be wrong. 
                        To show that this relation was not fixed before, we consider cases where it could have been fixed. We then show that such a configuration cannot exist. For this, we consdier another event $x$, and the following cases:
                        \begin{itemize}
                            \item Case where $\reln{x}{iwo}{w^j} \ \wedge \ \reln{x}{iwo}{w_i^j}$
                                %By Prop \ref{prop7}, $x$ is invalid.  
                            \item Case where $\reln{x}{iwo}{w^j} \ \wedge \ \reln{w_i^j}{iwo}{x}$
                                %pu union rf-1 is a cycle, thus violating coherence
                            \item Case where $\reln{w^j}{smo}{x} \ \wedge \ \reln{x}{iwo}{w_i^j}$
                                %There could exist events w' and x' above the read program ordered with w^j and x resp, with an implied write order. This could have been fixed before, thus giving us a relation with w^j and w_k^j, which could have also been fixed.
                                %By Def \ref{SymMemO} and Prop \ref{prop1}, we have $\reln{w'}{iwo}{x'}$. By Prop \ref{prop7} and Prop \ref{Wabv}, either $w'$ or $w^j$ is invalid, hence such a configuration cannot exist.           
                            \item Case where $\reln{x}{smo}{w^j} \ \wedge \ \reln{x}{iwo}{w_i^j}$
                                %There could exist events w' and x' above the read program ordered with w^j and x resp, with an implied write order. This could have been fixed before, thus giving us a relation with w^j and w_k^j, which could have also been fixed.
                                %By Def \ref{SymMemO} and Prop \ref{prop1}, we have $\reln{x'}{iwo}{w'}$. By Prop \ref{prop7} and Prop \ref{Wabv}, either $x'$ or $x$ is invalid, hence such a configuration cannot exist.
                            \item Case where $\reln{x}{smo}{w^j} \ \wedge \ \reln{w_i^j}{iwo}{x}$
                                %There could exist events w' and x' above the read program ordered with w^j and x resp, with an implied write order. This could have been fixed before, thus giving us a relation with w^j and w_k^j, which could have also been fixed.
                                %By Def \ref{SymMemO} and Prop \ref{prop1}, we have $\reln{x'}{iwo}{w'}$.
                                %But such a configuration cannot exist, because po union rf-1 is a cycle here, which violates coherence
                            \item Case where $\reln{w^j}{smo}{x} \ \wedge \ \reln{w_i^j}{iwo}{x}$
                                %There could exist events w' and x' above the read program ordered with w^j and x resp, with an implied write order. This could have been fixed before, thus giving us a relation with w^j and w_k^j, which could have also been fixed.
                                %By Def \ref{SymMemO} and Prop \ref{prop1}, we have $\reln{w'}{iwo}{x'}$. By Prop \ref{prop7} and Prop \ref{Wabv}, either $w'$ or $w^j$ is invalid, hence such a configuration cannot exist.
                        \end{itemize}
                
                \end{itemize}


            \paragraph{2. Write orders above read fixed , write orders below still need fixing.}
                Let us assume that the fixed set \textbf{above the read is the set of writes $w^j$}. Let us consider two threads $T_i$ and $T_k$ whose writes \textbf{below the read $w_i^l$ and $w_k^l$} is going to be fixed. Without loss of generality, let us conisder the symmetric memory order between these two threads to be of the form $\reln{w_i}{smo}{w_k}$.  By Prop \ref{prop5}, we then have the configuration as:
                \begin{align*}
                    \reln{w_i^j}{smo}{w_k^j} \ \wedge \ \reln{w_k^l}{iwo}{w_i^l}
                \end{align*}

                We divide our analysis into two cases 
                \begin{itemize}
                    \item There is an implied write order with $w_i^j$ and some $w^j$
                    
                        By Prop \ref{prop7} and Prop \ref{Wbel}, implied write orders between the two events can be of the form 
                        \begin{align*}
                            \reln{w^j}{iwo}{w_i^j} 
                        \end{align*}

                        By Def \ref{SymMemO} and Prop \ref{prop1}, we have
                        \begin{align*}
                            \reln{w^j}{smo}{w_k^j}
                        \end{align*}

                        By Prop \ref{prop3}, on swapping $T_i$ and $T_k$, we have 
                        \begin{align*}
                            \reln{w_i^l}{iwo}{w_k^l} \ \wedge \ \reln{w^j}{iwo}{w_k^j}
                        \end{align*}

                        THe above relations do not violate Prop \ref{prop1}, hence we can end this case here. 
                    
                    \item There is an implied write order with $w_k^j$ and some $w^j$
                        By Prop \ref{prop7} and Prop \ref{Wbel}, implied write orders between the two events can be of the form 
                        \begin{align*}
                            \reln{w_k^j}{iwo}{w^j} \\ 
                            \reln{w^j}{iwo}{w_k^j}  
                        \end{align*}

                        For the first case, by Def \ref{SymMemO} and Prop \ref{prop1}, we have
                        \begin{align*}
                            \reln{w_i^j}{smo}{w^j}
                        \end{align*}

                        By Prop \ref{prop3}, on swapping $T_i$ and $T_k$, we have 
                        \begin{align*}
                            \reln{w_i^l}{iwo}{w_k^l} \ \wedge \ \reln{w_i^j}{iwo}{w^j}
                        \end{align*}

                        The above relations do not violate Prop \ref{prop1}, hence we can end this case here. 

                        For the second case, by Prop \ref{prop3}, on swapping, we have 
                        \begin{align*}
                            \reln{w_i^l}{iwo}{w_k^l} \ \wedge \ \reln{w^j}{iwo}{w_i^j}
                        \end{align*}
                        The new relation $\reln{w^j}{iwo}{w_i^j}$ may be wrong. 
                        To show that this relation was not fixed before, we consider cases where it could have been fixed. We then show that such a configuration cannot exist. For this, we consdier another event $x$, and the following cases:

                        \begin{itemize}
                            \item Case where $\reln{x}{iwo}{w^j} \ \wedge \ \reln{x}{iwo}{w_i^j}$
                                %By Prop \ref{prop7}, $x$ is invalid. 
                            \item Case where $\reln{w^j}{smo}{x} \ \wedge \ \reln{x}{iwo}{w_i^j}$
                                %There could exist events w' and x' above the read program ordered with w^j and x resp, with an implied write order. This could have been fixed before, thus giving us a relation with w^j and w_i^j, which could have also been fixed.
                                %By Def \ref{SymMemO} and Prop \ref{prop1}, we have $\reln{w'}{iwo}{x'}$.
                                %By Prop \ref{prop7} and Prop \ref{Wabv}, either $w'$ or $w^j$ is invalid. 
                            \item Case where $\reln{x}{smo}{w^j} \ \wedge \ \reln{x}{iwo}{w_i^j}$
                                %There could exist events w' and x' above the read program ordered with w^j and x resp, with an implied write order. This could have been fixed before, thus giving us a relation with w^j and w_i^j, which could have also been fixed.
                                %By Def \ref{SymMemO} and Prop \ref{prop1}, we have $\reln{x'}{iwo}{w'}$.
                                %By Prop \ref{prop7} and Prop \ref{Wabv}, either $x'$ or $x$ is invalid. 
                        \end{itemize}

                \end{itemize}

            
            \paragraph{3. Write orders below read fixed , write orders below still need fixing.}
                Let us assume that the fixed set \textbf{below the read is the set of writes $w^j$}. Let us consider two threads $T_i$ and $T_k$ whose writes \textbf{below the read $w_i^l$ and $w_k^l$} is going to be fixed. Without loss of generality, let us conisder the symmetric memory order between these two threads to be of the form $\reln{w_i}{smo}{w_k}$.  By Prop \ref{prop5}, we then have the configuration as:
                \begin{align*}
                    \reln{w_i^j}{smo}{w_k^j} \ \wedge \ \reln{w_k^l}{iwo}{w_i^l}
                \end{align*}
                
                We divide our analysis into two cases 
                \begin{itemize}
                    \item There is an implied write order with $w_i^j$ and some $w^j$
                        
                        By Prop \ref{prop8} and Prop \ref{Wbel}, implied write orders between the two events can be of the form 
                        \begin{align*}
                            \reln{w_i^j}{iwo}{w^j} 
                        \end{align*}

                        By Prop \ref{prop3}, on swapping, we have 
                        \begin{align*}
                            \reln{w_i^l}{iwo}{w_k^l} \ \wedge \ \reln{w_k^j}{iwo}{w^j}
                        \end{align*}
                        The new relation $\reln{w_k^j}{iwo}{w^j}$ may be wrong. 
                        To show that this relation was not fixed before, we consider cases where it could have been fixed. We then show that such a configuration cannot exist. For this, we consdier another event $x$, and the following cases:

                        \begin{itemize}
                            \item Case where $\reln{w^j}{iwo}{x} \ \wedge \ \reln{x}{iwo}{w_k^j}$
                                %po union rf cycle, thus violating coherence.
                            \item Case where $\reln{w^j}{iwo}{x} \ \wedge \ \reln{w_k^j}{iwo}{x}$
                                %By Prop \ref{prop8}, $x$ is invalid. 
                            \item Case where $\reln{w^j}{smo}{x} \ \wedge \ \reln{x}{iwo}{w_k^j}$
                                %There could exist events w' and x' above the read program ordered with w^j and x resp, with an implied write order. This could have been fixed before, thus giving us a relation with w^j and w_i^j, which could have also been fixed.
                                %By Def \ref{SymMemO} and Prop \ref{prop1}, we have $\reln{w'}{iwo}{x'}$.
                                %By Prop \ref{Wbel} and Prop \ref{prop7}, either $w'$ or $w^j$ are invalid. 

                                %There could exist events w' and x' below the read program ordered with w^j and x resp, with an implied write order. This could have been fixed before, thus giving us a relation with w^j and w_i^j, which could have also been fixed.
                                %By Def \ref{SymMemO} and Prop \ref{prop1}, we have $\reln{w'}{iwo}{x'}$.
                                %In this case, we have po union rf as a cycle, thus violating coherence. 

                            \item Case where $\reln{x}{smo}{w^j} \ \wedge \ \reln{x}{iwo}{w_k^j}$
                                %There could exist events w' and x' above the read program ordered with w^j and x resp, with an implied write order. This could have been fixed before, thus giving us a relation with w^j and w_i^j, which could have also been fixed.
                                %By Def \ref{SymMemO} and Prop \ref{prop1}, we have $\reln{x'}{iwo}{w'}$.
                                
                                %The only way to get a configuration with $\reln{w_k^j}{iwo}{w^j}$ before would be to swap $T_x$ and $T'$ first, followed by swapping $T'$ and $T_k$. 
                                %But this swapping order is not possible, as the "fixing" process we have is to fix implied write orders above read first and then below. In that sense, the de-swapping here can only be implied write orders below read and then above read. 
                                %Since this is not possible in the above configuration, we can conclude that $\reln{w_k^j}{iwo}{w^j}$ did not exist before and hence could not have been fixed.
                                
                                \critic{blue}{The reader should note that though it was possible for the correct relation $\reln{w^j}{iwo}{w_k^j}$ to exist, it wasn't something that was "fixed" by swapping. This is different from those implied write orders which were fixed by swapping thread identities. (Discuss with Viktor on this)}

                                %There could exist events w' and x' below the read program ordered with w^j and x resp, with an implied write order. This could have been fixed before, thus giving us a relation with w^j and w_i^j, which could have also been fixed.
                                %By Def \ref{SymMemO} and Prop \ref{prop1}, we have $\reln{x'}{iwo}{w'}$.
                                %By Prop \ref{Wbel} and Prop \ref{prop8}, either $w'$ or $w^j$ are invalid. 

                            \item Case where $\reln{w^j}{smo}{x} \ \wedge \ \reln{w_k^j}{iwo}{x}$
                                %There could exist events w' and x' above the read program ordered with w^j and x resp, with an implied write order. This could have been fixed before, thus giving us a relation with w^j and w_i^j, which could have also been fixed.
                                %By Def \ref{SymMemO} and Prop \ref{prop1}, we have $\reln{w'}{iwo}{x'}$.
                                %By Prop \ref{Wbel} and Prop \ref{prop7}, either $w'$ or $w^j$ are invalid. 

                                %There could exist events w' and x' below the read program ordered with w^j and x resp, with an implied write order. This could have been fixed before, thus giving us a relation with w^j and w_i^j, which could have also been fixed.
                                %By Def \ref{SymMemO} and Prop \ref{prop1}, we have $\reln{w'}{iwo}{x'}$.
                                %By Prop \ref{prop8} nad Prop \ref{Wbel}, either $x'$ or $x$ are invalid. 

                            \item Case where $\reln{x}{smo}{w^j} \ \wedge \ \reln{w_k^j}{iwo}{x}$
                                %There could exist events w' and x' above the read program ordered with w^j and x resp, with an implied write order. This could have been fixed before, thus giving us a relation with w^j and w_i^j, which could have also been fixed.
                                %By Def \ref{SymMemO} and Prop \ref{prop1}, we have $\reln{x'}{iwo}{w'}$.
                                %By Prop \ref{Wbel} and Prop \ref{prop7}, either $x'$ or $x$ are invalid. 

                                %There could exist events w' and x' below the read program ordered with w^j and x resp, with an implied write order. This could have been fixed before, thus giving us a relation with w^j and w_i^j, which could have also been fixed.
                                %By Def \ref{SymMemO} and Prop \ref{prop1}, we have $\reln{x'}{iwo}{w'}$.
                                %By Prop \ref{Wbel} and Prop \ref{prop8}, either $w'$ or $w^j$ are invalid. 

                        \end{itemize}
                        
                    \item There is an implied write order with $w_k^j$ and some $w^j$
                    
                        By Prop \ref{prop8} and Prop \ref{Wbel}, implied write orders between the two events can be of the form 
                        \begin{align*}
                            \reln{w_k^j}{iwo}{w^j} \\
                            \reln{w^j}{iwo}{w_k^j} 
                        \end{align*}

                        For the first case, by Def \ref{SymMemO} and Prop \ref{prop1}, we have
                        \begin{align*}
                            \reln{w_i^j}{smo}{w^j}
                        \end{align*}

                        By Prop \ref{prop3}, on swapping $T_i$ and $T_k$, we have 
                        \begin{align*}
                            \reln{w_i^l}{iwo}{w_k^l} \ \wedge \ \reln{w_i^j}{iwo}{w^j}
                        \end{align*}

                        The above relations do not violate Prop \ref{prop1}, hence we can end this case here. 

                        For the second case, by Prop \ref{prop3}, on swapping, we have 
                        \begin{align*}
                            \reln{w_i^l}{iwo}{w_k^l} \ \wedge \ \reln{w^j}{iwo}{w_i^j}
                        \end{align*}
                        The new relation $\reln{w^j}{iwo}{w_i^j}$ may be wrong. 
                        To show that this relation was not fixed before, we consider cases where it could have been fixed. We then show that such a configuration cannot exist. For this, we consdier another event $x$, and the following cases:

                        \begin{itemize}
                            \item Case where $\reln{x}{iwo}{w^j} \ \wedge \ \reln{w_i^j}{iwo}{x}$
                                % po union rf cycle, thus violating coherence
                            \item Case where $\reln{w^j}{iwo}{x} \ \wedge \ \reln{w_i^j}{iwo}{x}$
                                % By Prop \ref{prop8}, $x$ is invalid. 
                            \item Case where $\reln{x}{smo}{w^j} \ \wedge \ \reln{w_i^j}{iwo}{x}$
                                %There could exist events w' and x' above the read program ordered with w^j and x resp, with an implied write order. This could have been fixed before, thus giving us a relation with w^j and w_i^j, which could have also been fixed.
                                %By Def \ref{SymMemO} and Prop \ref{prop1}, we have $\reln{x'}{iwo}{w'}$.
                                %By Prop \ref{Wbel} and Prop \ref{prop7}, either $x'$ or $x$ are invalid.
                                
                                %There could exist events w' and x' below the read program ordered with w^j and x resp, with an implied write order. This could have been fixed before, thus giving us a relation with w^j and w_i^j, which could have also been fixed.
                                %By Def \ref{SymMemO} and Prop \ref{prop1}, we have $\reln{x'}{iwo}{w'}$.
                                %We have po union rf cycle, thus violating coherence. 
                                
                            \item Case where $\reln{w^j}{smo}{x} \ \wedge \ \reln{w_i^j}{iwo}{x}$
                                %There could exist events w' and x' above the read program ordered with w^j and x resp, with an implied write order. This could have been fixed before, thus giving us a relation with w^j and w_i^j, which could have also been fixed.
                                %By Def \ref{SymMemO} and Prop \ref{prop1}, we have $\reln{w'}{iwo}{x'}$.

                                %The only way to get a configuration with $\reln{w^j}{iwo}{w_i^j}$ before would be to swap $T_x$ and $T'$ first, followed by swapping $T'$ and $T_i$. 
                                %But this swapping order is not possible, as the "fixing" process we have is to fix implied write orders above read first and then below. The de-swapping here can only be implied write orders below read and then above read. 
                                %Since this is not possible in the above configuration, we can conclude that $\reln{w^j}{iwo}{w_i^j}$ did not exist before and hence could not have been fixed.
                                %TO BE DONE LATER
                                \critic{blue}{The reader should note that though it was possible for the correct relation $\reln{w^j}{iwo}{w_k^j}$ to exist, it wasn't something that was "fixed" by swapping. This is different from those implied write orders which were fixed by swapping thread identities. (Discuss with Viktor on this)}

                                %There could exist events w' and x' below the read program ordered with w^j and x resp, with an implied write order. This could have been fixed before, thus giving us a relation with w^j and w_i^j, which could have also been fixed.
                                %By Def \ref{SymMemO} and Prop \ref{prop1}, we have $\reln{w'}{iwo}{x'}$.
                                %By Prop \ref{Wbel} and Prop \ref{prop8}, either $x'$ or $x$ are invalid. 

                        \end{itemize}

                \end{itemize}
            
            We still will have another case for any new implied write order introduced, (Consider this cases in the end. We need to divide it into two cases.)
            \paragraph{4. Write orders below and above read fixed , write orders above need fixing.} 
                
                Let us assume that the fixed set \textbf{below the read is the set of writes $w^j$}. Let us consider two threads $T_i$ and $T_k$ whose writes \textbf{above the read $w_i^l$ and $w_k^l$} is going to be fixed. Without loss of generality, let us conisder the symmetric memory order between these two threads to be of the form $\reln{w_i}{smo}{w_k}$.  By Prop \ref{prop5}, we then have the configuration as:
                \begin{align*}
                    \reln{w_i^j}{smo}{w_k^j} \ \wedge \ \reln{w_k^l}{iwo}{w_i^l}
                \end{align*}

                We divide our analysis into two cases 
                \begin{itemize}
                    \item There is an implied write order with $w_i^j$ and some $w^j$
                        By Prop \ref{prop8} and Prop \ref{Wabv}, implied write orders between the two events can be of the form 
                        \begin{align*}
                            \reln{w^j}{iwo}{w_i^j} \\
                            \reln{w_i^j}{iwo}{w^j} 
                        \end{align*}

                        For the first case by Def \ref{SymMemO} and Prop \ref{prop1}, we have
                        \begin{align*}
                            \reln{w^j}{smo}{w_k^j}
                        \end{align*}

                        By Prop \ref{prop3}, on swapping $T_i$ and $T_k$, we have 
                        \begin{align*}
                            \reln{w_i^l}{iwo}{w_k^l} \ \wedge \ \reln{w^j}{iwo}{w_k^j}
                        \end{align*}

                        The above relations do not violate Prop \ref{prop1}, hence we can end this case here. 

                        For the second case, by Prop \ref{prop3}, on swapping, we have 
                        \begin{align*}
                            \reln{w_i^l}{iwo}{w_k^l} \ \wedge \ \reln{w_k^j}{iwo}{w^j}
                        \end{align*}

                        The new relation $\reln{w_k^j}{iwo}{w^j}$ may be wrong. 
                        To show that this relation was not fixed before, we consider cases where it could have been fixed. We then show that such a configuration cannot exist. For this, we consdier another event $x$, and the following cases:

                        \begin{itemize}
                            \item Case where $\reln{w^j}{iwo}{x} \ \wedge \ \reln{w_k^j}{iwo}{x}$
                                %By Prop \ref{prop8}, $x$ is invalid.


                            \item Case where $\reln{w^j}{smo}{x} \ \wedge \ \reln{w_k^j}{iwo}{x}$
                                %There could exist events w' and x' above the read program ordered with w^j and x resp, with an implied write order. This could have been fixed before, thus giving us a relation with w^j and w_i^j, which could have also been fixed.
                                %By Def \ref{SymMemO} and Prop \ref{prop1}, we have $\reln{w'}{iwo}{x'}$.
                                %By Prop \ref{Wbel}, Prop \ref{Wabv} and Prop \ref{prop7}, either $w'$ or $w^j$ are invalid. 

                                %There could exist events w' and x' below the read program ordered with w^j and x resp, with an implied write order. This could have been fixed before, thus giving us a relation with w^j and w_i^j, which could have also been fixed.
                                %By Def \ref{SymMemO} and Prop \ref{prop1}, we have $\reln{w'}{iwo}{x'}$.
                                %By Prop \ref{Wbel} and Prop \ref{prop7}, either $x'$ or $x$ are invalid. 

                            \item Case where $\reln{x}{smo}{w^j} \ \wedge \ \reln{w_k^j}{iwo}{x}$
                                %There could exist events w' and x' above the read program ordered with w^j and x resp, with an implied write order. This could have been fixed before, thus giving us a relation with w^j and w_i^j, which could have also been fixed.
                                %By Def \ref{SymMemO} and Prop \ref{prop1}, we have $\reln{x'}{iwo}{w'}$.
                                %By Prop \ref{Wbel}, Prop \ref{Wabv} and Prop \ref{prop7}, either $x'$ or $x$ are invalid.

                                %There could exist events w' and x' below the read program ordered with w^j and x resp, with an implied write order. This could have been fixed before, thus giving us a relation with w^j and w_i^j, which could have also been fixed.
                                %By Def \ref{SymMemO} and Prop \ref{prop1}, we have $\reln{x'}{iwo}{w'}$.
                                %By Prop \ref{Wbel} and Prop \ref{prop7}, either $w'$ or $w^j$ are invalid. 
                            
                        \end{itemize}
                        
                    \item There is an implied write order with $w_k^j$ and some $w^j$
                        By Prop \ref{prop8} and Prop \ref{Wabv}, implied write orders between the two events can be of the form 
                        \begin{align*}
                            \reln{w_k^j}{iwo}{w^j}  
                        \end{align*}

                        For the first case by Def \ref{SymMemO} and Prop \ref{prop1}, we have
                        \begin{align*}
                            \reln{w_i^j}{smo}{w^j}
                        \end{align*}

                        By Prop \ref{prop3}, on swapping $T_i$ and $T_k$, we have 
                        \begin{align*}
                            \reln{w_i^l}{iwo}{w_k^l} \ \wedge \ \reln{w_i^j}{iwo}{w^j}
                        \end{align*}

                        The above relations do not violate Prop \ref{prop1}, hence we can end this case here. 
                \end{itemize}

                %-----------------------------------------------------------------------------------------------------------------------

                Let us assume that the fixed set \textbf{above the read is the set of writes $w^j$}. Let us consider two threads $T_i$ and $T_k$ whose writes \textbf{above the read $w_i^l$ and $w_k^l$} is going to be fixed. Without loss of generality, let us conisder the symmetric memory order between these two threads to be of the form $\reln{w_i}{smo}{w_k}$.  By Prop \ref{prop5}, we then have the configuration as:
                \begin{align*}
                    \reln{w_i^j}{smo}{w_k^j} \ \wedge \ \reln{w_k^l}{iwo}{w_i^l}
                \end{align*}

                We divide our analysis into two cases 
                \begin{itemize}
                    \item There is an implied write order with $w_i^j$ and some $w^j$

                        By Prop \ref{prop7} and Prop \ref{Wabv}, implied write orders between the two events can be of the form 
                        \begin{align*}
                            \reln{w^j}{iwo}{w_i^j} \\
                            \reln{w_i^j}{iwo}{w^j} 
                        \end{align*}

                        For the first case by Def \ref{SymMemO} and Prop \ref{prop1}, we have
                        \begin{align*}
                            \reln{w^j}{smo}{w_k^j}
                        \end{align*}

                        By Prop \ref{prop3}, on swapping $T_i$ and $T_k$, we have 
                        \begin{align*}
                            \reln{w_i^l}{iwo}{w_k^l} \ \wedge \ \reln{w^j}{iwo}{w_k^j}
                        \end{align*}

                        The above relations do not violate Prop \ref{prop1}, hence we can end this case here. 

                        For the second case, by Prop \ref{prop3}, on swapping, we have 
                        \begin{align*}
                            \reln{w_i^l}{iwo}{w_k^l} \ \wedge \ \reln{w_k^j}{iwo}{w^j}
                        \end{align*}

                        The new relation $\reln{w_k^j}{iwo}{w^j}$ may be wrong. 
                        To show that this relation was not fixed before, we consider cases where it could have been fixed. We then show that such a configuration cannot exist. Since most of these cases have already been covered, we will only address those cases which aren't; i.e. those where certain implied write orders between writes below the read have been "fixed". For this, we consdier another event $x$, and the following cases:

                        \begin{itemize}
                            \item Case where $\reln{w^j}{smo}{x} \ \wedge \ \reln{x}{iwo}{w_k^j}$
                                %There could exist events w' and x' below the read program ordered with w^j and x resp, with an implied write order. This could have been fixed before, thus giving us a relation with w^j and w_i^j, which could have also been fixed.
                                %By Def \ref{SymMemO} and Prop \ref{prop1}, we have $\reln{w'}{iwo}{x'}$

                                %The only way to get a configuration with $\reln{w_k^j}{iwo}{w^j}$ before would be to swap $T_x$ and $T'$ first, followed by swapping $T'$ and $T_i$. 
                                %On swapping $T_x$ and $T'$, we have the following configuration 
                                \begin{align*}
                                    \reln{x'}{iwo}{w'} \ \wedge \ \reln{w_k^l}{iwo}{w_i^l} \ \wedge \ \reln{w^j}{iwo}{w_k^j} \ \wedge \ \reln{x}{iwo}{w_i^j} 
                                \end{align*}
                                %If this was the original configuration of execution, we can note that $\reln{w_k^l}{iwo}{w_i^l}$ would be fixed first as it is between writes above the read, in contrast to $\reln{x'}{iwo}{w'}$. Hence, it is not possible to have gotten our intended execution graph from this configuration. Thus, we can conclude that $\reln{w_k^j}{iwo}{w^j}$ could not have been fixed before.  

                                %Since this is not possible in the above configuration, we can conclude that $\reln{w^j}{iwo}{w_i^j}$ did not exist before and hence could not have been fixed.
                                %TO BE ADDRESSED SEPARATELY

                            \item Case where $\reln{x}{smo}{w^j} \ \wedge \ \reln{x}{iwo}{w_k^j}$
                                %There could exist events w' and x' below the read program ordered with w^j and x resp, with an implied write order. This could have been fixed before, thus giving us a relation with w^j and w_i^j, which could have also been fixed.
                                %By Def \ref{SymMemO} and Prop \ref{prop1}, we have $\reln{x'}{iwo}{w'}$
                                %By Prop \ref{prop8} and Prop \ref{Wbel}, either $w'$ or $w^j$ is invalid.
                        \end{itemize}



                    \item There is an implied write order with $w_k^j$ and some $w^j$

                        By Prop \ref{prop7} and Prop \ref{Wabv}, implied write orders between the two events can be of the form 
                        \begin{align*}
                            \reln{w^j}{iwo}{w_k^j}  
                        \end{align*}

                        By Prop \ref{prop3}, on swapping, we have 
                        \begin{align*}
                            \reln{w_i^l}{iwo}{w_k^l} \ \wedge \ \reln{w^j}{iwo}{w_i^j}
                        \end{align*}

                        The new relation $\reln{w^j}{iwo}{w_i^j}$ may be wrong. 
                        To show that this relation was not fixed before, we consider cases where it could have been fixed. We then show that such a configuration cannot exist. Since most of these cases have already been covered, we will only address those cases which aren't; i.e. those where certain implied write orders between writes below the read have been "fixed". For this, we consdier another event $x$, and the following cases:

                        \begin{itemize}
                            \item Case where $\reln{w^j}{smo}{x} \ \wedge \ \reln{w_i^j}{iwo}{x}$
                                %There could exist events w' and x' below the read program ordered with w^j and x resp, with an implied write order. This could have been fixed before, thus giving us a relation with w^j and w_i^j, which could have also been fixed.
                                %By Def \ref{SymMemO} and Prop \ref{prop1}, we have $\reln{w'}{iwo}{x'}$
                                
                                %TO BE ADDRESSED SEPARATELY
                                %The only way to get a configuration with $\reln{w^j}{iwo}{w_i^j}$ before would be to swap $T_x$ and $T'$ first, followed by swapping $T'$ and $T_i$. 
                                %On swapping $T_x$ and $T'$, we have the following configuration 
                                \begin{align*}
                                    \reln{x'}{iwo}{w'} \ \wedge \ \reln{w_k^l}{iwo}{w_i^l} \ \wedge \ \reln{w_i^j}{iwo}{w^j} \ \wedge \ \reln{x}{iwo}{w_i^j} 
                                \end{align*}
                                %If this was the original configuration of execution, we can note that $\reln{w_k^l}{iwo}{w_i^l}$ would be fixed first as it is between writes above the read, in contrast to $\reln{x'}{iwo}{w'}$ which is between writes below the read. Hence, it is not possible to have gotten our intended execution graph from this configuration. Thus, we can conclude that $\reln{w^j}{iwo}{w_i^j}$ could not have been fixed before.

                            \item Case where $\reln{x}{smo}{w^j} \ \wedge \ \reln{w_i^j}{iwo}{x}$
                                %There could exist events w' and x' below the read program ordered with w^j and x resp, with an implied write order. This could have been fixed before, thus giving us a relation with w^j and w_i^j, which could have also been fixed.
                                %By Def \ref{SymMemO} and Prop \ref{prop1}, we have $\reln{x'}{iwo}{w'}$
                                %By Prop \ref{prop8} and Prop \ref{Wbel}, either $w'$ or $w^j$ is invalid.

                            \item Case where $\reln{w^j}{smo}{x} \ \wedge \ \reln{x}{iwo}{w_i^j}$
                                %There could exist events w' and x' below the read program ordered with w^j and x resp, with an implied write order. This could have been fixed before, thus giving us a relation with w^j and w_i^j, which could have also been fixed.
                                %By Def \ref{SymMemO} and Prop \ref{prop1}, we have $\reln{w'}{iwo}{x'}$
                                %By Prop \ref{prop8} and Prop \ref{Wbel}, either $x'$ or $x$ is invalid.

                            \item Case where $\reln{x}{smo}{w^j} \ \wedge \ \reln{x}{iwo}{w_i^j}$
                                %There could exist events w' and x' below the read program ordered with w^j and x resp, with an implied write order. This could have been fixed before, thus giving us a relation with w^j and w_i^j, which could have also been fixed.
                                %By Def \ref{SymMemO} and Prop \ref{prop1}, we have $\reln{x'}{iwo}{w'}$
                                %By Prop \ref{prop8} and Prop \ref{Wbel}, either $w'$ or $w^j$ is invalid.

                        \end{itemize}
                \end{itemize}

                In summary, we can conclude that new implied write orders among other set of writes introduced while fixing an implied write order, if wrong cannot have been resolved earlier, and thus, by Lemma \ref{lemma1} can be fixed and will remain fixed.

        \end{proof}

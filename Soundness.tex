\subsection{Soundness of the rules above}

    To prove soundness, we first define the following: 

    \begin{definition}{Implied Write Order(\emph{iwo})}
        Binary relation between any two \emph{distinct} writes, derived through the following two sequential conposition:  
        \begin{align*}
            w_i^j;po;rf^{-1};w_k^j \\
            w_i^j;rf;po;w_k^j 
        \end{align*}
    \end{definition}

%---------------------------------------------------------------------------------------------------------------------------------    

    \begin{property}{Simplified irreflexivity rule}
        
        The irreflexivity constraint rule is equivalent to the following irreflexivity condition 
        \begin{align}
            smo;iwo
        \end{align}
    \end{property}

    \begin{proof}
        Expanding for implied write order as per the definition, gives us the following two sequential compositions. 
        \begin{align}
            smo;w_i^j;po;rf^{-1};w_k^j \\
            smo;w_i^j;rf;po;w_k^j 
        \end{align}
        From the definiton of symmetric order, the above can be simplified to
        \begin{align}
            smo;po;rf^{-1} \\
            smo;rf;po 
        \end{align}
        Hence, proving our property. 
    \end{proof}

%-----------------------------------------------------------------------------------------------------------------------------------

    \begin{property}
        No write order is implied when a read reads from its own thread's write
    \end{property}

    \begin{proof}
        If the read is from its own thread's write, then we can infer that $i=k$ in both the sequential compositions. Hence  
        \begin{align*}
            w_i^j;rf;po;w_i^j \\
            w_i^j;po;rf^{-1};w_i^j
        \end{align*} 
        which gives us $\reln{w_i^j}{iwo}{w_i^j}$.
        Since implied write orders are only between distinct writes, the property is proven.  
    \end{proof}
        
%-------------------------------------------------------------------------------------------------------------------------------------

    \begin{property}
        Implied write orders between two symmetric threads are reversed when they are swapped.
    \end{property}
        
    \begin{proof}
        Considering first sequential composition, i.e. $w_i^j;rf;po;w_k^j$, expanding gives us the following binary relations involved:
        \begin{align}
            \reln{w_i^j}{rf}{r_k} \\
            \reln{r_k}{po}{w_k^j}
        \end{align}
        Swapping thread identities involves swapping the indices $i$ and $k$ for each event, thus giving us 
        \begin{align}
            \reln{w_k^j}{rf}{r_i} \\
            \reln{r_i}{po}{w_i^j}
        \end{align}
        Through sequential composition of the above, we get $w_k^j;rf;po;w_i^j$, which by definition is $\reln{w_k^j}{iwo}{w_i^j}$.

        The argument is symmetric for the second sequential composition. 
    \end{proof}
        
%-------------------------------------------------------------------------------------------------------------------------------------

    \begin{property}
        There are at most two implied write orders between writes of two threads.
    \end{property}
        
    \begin{proof}
        Consider two threads $T_i$ and $T_k$. Suppose we have one implied write order between one of their writes, i.e. 
        \begin{align}
            \reln{w_i^j}{iwo}{w_k^j}.    
        \end{align}
        Expanding as per the first sequential composition gives us 
        \begin{align}
            w_i^j;po;rf^{-1};w_k^j
        \end{align}
        which also indicates a $\stck{_{rf}}$ with $T_i$'s read and that the writes involved in the composition are above the respective reads.

        Now suppose we have an implied write order between another set of writes, i.e.
        \begin{align}
            \reln{w_i^l}{iwo}{w_k^l}.    
        \end{align}
        Expanding as per the first sequential composition of implied write order is not possible as $\stck{_{rf}}$ is functional. Hence, using the second we have
        \begin{align}
            w_i^j;rf;po;w_k^j 
        \end{align}  
        which also indicates a $\stck{_{rf}}$ with $T_i$'s read that the writes involved in the composition are below the respective reads.

        Since both reads are now involved in a $\stck{_{rf}}$ relation, and since this relation is functional, we cannot have any more implied write orders between $T_i$ and $T_j$, thus verifying our property. 

        \critic{blue}{Better written in contrast to previous argument.}
        \critic{red}{However, is it necessary to show by contradiction?}
    \end{proof}
        
%--------------------------------------------------------------------------------------------------------------------------------------

    \begin{property}
        Implied write ordes between two threads are either all compliant with $stck{_{smo}}$ or they are all not
    \end{property}

    \begin{proof}
        
        If each read reads from its own write, we have no implied write order established, thus maintaining the property. 

        For cases where implied write orders are established, without loss of generality, let us consider one between writes above the read are compliant with $stck{_{smo}}$:
        \begin{align}
            \reln{w_i^j}{iwo}{w_k^j} \ \wedge \ \reln{w_i^j}{smo}{w_k^j}
        \end{align}
        The other set of implied write order, if established can only be between writes below the read. Suppose we have such an order but not compliant with $stck{_{smo}}$:
        \begin{align}
            \reln{w_k^l}{iwo}{w_i^l} \ \wedge \ \reln{w_i^l}{smo}{w_k^l}
        \end{align}
        Upon expanding using the second sequential composition (because writes are below the read), we get
        \begin{align}
            w_k^l;rf;po;w_i^l 
        \end{align}
        But this implies another $\stck{_{rf}}$ relation with $T_i$'s read, which violates the functional property of it. Hence we can only have an implied write order compliant with $stck{_{smo}}$.
        \begin{align}
            \reln{w_i^l}{iwo}{w_k^l} \ \wedge \ \reln{w_i^l}{smo}{w_k^l}
        \end{align}

        \critic{red}{Not sure if we need to show that the compliant relation also holds as it only brings an rf realtion with $T_k$'s read, which wanst established before.}

        The opposite case would make both the implied write orders requiring to not be compliant, thus by symmetry completing our proof. 
    \end{proof}

%-------------------------------------------------------------------------------------------------------------------------------------- 
        
    \begin{property}
        Implied write orders are acyclic
    \end{property}

    \begin{proof}
        %By Contradiction 
        Suppose a cycle exists. Then without loss of generality, we can consider the cycle composed of 3 writes.
        \begin{align}
            \reln{w_i^j}{iwo}{w_k^j} \ \wedge \reln{w_k^j}{iwo}{w_l^j} \ \wedge \reln{w_l^j}{iwo}{w_i^j}  
        \end{align}
        
        If these writes are above the read, then we have the following relations that result in the above cycle.
        \begin{align}
            \reln{w_i^j}{po}{r_i} \ \wedge \ \reln{r_i}{rf^{-1}}{w_k^j} \\
            \reln{w_k^j}{po}{r_k} \ \wedge \ \reln{r_k}{rf^{-1}}{w_l^j} \\
            \reln{w_l^j}{po}{r_l} \ \wedge \ \reln{r_l}{rf^{-1}}{w_i^j} 
        \end{align}
        The above relations form a cycle thus violating $po \cup rf^{-1} \ \text{acyclic}$ rule for coherence and hence violating coherence.
        
        If these writes are below the read, then we have the following relations that result in the above cycle.
        \begin{align}
            \reln{w_i^j}{po}{r_i} \ \wedge \ \reln{r_i}{rf}{w_k^j} \\
            \reln{w_k^j}{po}{r_k} \ \wedge \ \reln{r_k}{rf}{w_l^j} \\
            \reln{w_l^j}{po}{r_l} \ \wedge \ \reln{r_l}{rf}{w_i^j} 
        \end{align}
        The above relations form a cycle thus violating $po \cup rf \ \text{acyclic}$ rule for coherence and hence violating coherence.

        Because both cases violate coherence, we conclude that $\stck{_{iwo}}$ must be acylic. 

        \critic{blue}{A bit better written compared to the previous proofs}
    \end{proof}

%---------------------------------------------------------------------------------------------------------------------------------------

    \critic{red}{Need to also prove the properties of incmoing or outgoing write orders.}

    \subsubsection{Soundness}

        To prove that our rules our sound, we show that for every implied write order such that $smo;iwo$ is reflexive, we can swap the corresponding thread identities to reverse the write order between them, thus respectiing our irreflexivity constraint. However, we must ensure that once write orders are "fixed" in this fashion, they remain fixed, i.e. the relation cannot appear again to be wrong. Once we show this, we also need to show that this holds in general given multiple sets of equal writes. 
        
        \paragraph{Part1}
            For a given set of equal writes, once an implied write order is fixed, it remains fixed. 
            (A more formal statement required)
        \begin{proof}
            %Proof by contradiction 
            Suppose we have one implied write order which is correct / fixed between writes $w_i^j$ and $w_k^j$.
            \begin{align}
                \reln{w_i^j}{iwo}{w_k^j}
            \end{align}
            Without loss of generality, suppose these writes are above the read. We then divide our concern into two parts, one with implied write orders with $w_i^j$ which are wrong and the second with those of $w_k^j$.

            Case1: 
            Because $w_i^j$ is a write above read and $\reln{w_i^j}{iwo}{w_k^j}$, other implied write orders with $w_i^j$ can be of the form:
            \begin{align}
                \reln{w_m^j}{iwo}{w_i^j}
            \end{align}
            If this implied order is wrong, swapping thread identities will  give us the following relations 
            \begin{align}
                \reln{w_i^j}{iwo}{w_m^j} \ \wedge \ \reln{w_m^j}{iwo}{w_k^j}  
            \end{align}  
            Here, there is no implied write order relation between $w_i^j$ and $w_k^j$, hence we can consider them as being fixed. 
            If the implied order between $w_m^j$ and $w_k^j$ is wrong, swapping it will result in the following relations
            \begin{align}
                \reln{w_i^j}{iwo}{w_k^j} \ \wedge \ \reln{w_k^j}{iwo}{w_m^j}  
            \end{align}  
            Here, the implied write order between $w_i^j$ and $w_k^j$ remains the same. Thus, for this case, we can conclude that the relation remains "fixed".

            Case2: 
            To be done.
        \end{proof}


        

    \paragraph{Case for three or more threads}

        Method 1: For writes above and below (The general case of soundness)
        \begin{itemize}
            \item Proof by contradiction 
            \item Suppose all write orders above reads are sorted as per our need. 
            \item Then for writes below, firstly, if an implied write order is not respected, then either the writes above also not respected or it is a free order. 
            \item If it is not respected, this contradicts our assumption.
            \item If it is free, then we can freely swap the two threads, fixing the implied wrtie order of the write below the read. 
            \item The above writes will still be in order, because the above write order between those two threads swapped was free, and smo is total w.r.t the equal writes of each symmetric thread.  
        \end{itemize}

        \critic{blue}{The above two acyclic parts of coherence is only for events inovlved in one memory operation. The respective generalized forms would be restrictions of Load Buffering or Out of Thin Air values, which is not of our concern when we consider just writes and read to same memory.}
    

        Part2: The case for only one set of equal writes 
        \begin{itemize}
            \item Show that once an implied write order between two writes is fixed, it will never be un-fixed again.
            \item To show this, we can keep the implied write order transitive. 
            \item Go by assuming an order to exist between two writes which is correct. 
            \item Consider another implied order between one of these wrties and another outside which is not correct. 
        \end{itemize}
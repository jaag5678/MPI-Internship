\subsection{Soundness of the rules above}

    To prove soundness, we first define the following: 

    \begin{definition}{Implied Write Order(\emph{iwo})}
        Binary relation between any two \emph{distinct} writes, derived through the following two sequential conposition:  
        \begin{align*}
            w_i^j;po;rf^{-1};w_k^j \\
            w_i^j;rf;po;w_k^j 
        \end{align*}
    \end{definition}

%---------------------------------------------------------------------------------------------------------------------------------    

    \begin{property}{Simplified irreflexivity rule}
        
        The irreflexivity constraint rule is equivalent to the following irreflexivity condition 
        \begin{align}
            smo;iwo
        \end{align}
    \end{property}

    \begin{proof}
        Expanding for implied write order as per the definition, gives us the following two sequential compositions. 
        \begin{align}
            smo;w_i^j;po;rf^{-1};w_k^j \\
            smo;w_i^j;rf;po;w_k^j 
        \end{align}
        From the definiton of symmetric order, the above can be simplified to
        \begin{align}
            smo;po;rf^{-1} \\
            smo;rf;po 
        \end{align}
        Hence, proving our property. 
    \end{proof}

%-----------------------------------------------------------------------------------------------------------------------------------

    \begin{property}
        No write order is implied when a read reads from its own thread's write
    \end{property}

    \begin{proof}
        If the read is from its own thread's write, then we can infer that $i=k$ in both the sequential compositions. Hence  
        \begin{align*}
            w_i^j;rf;po;w_i^j \\
            w_i^j;po;rf^{-1};w_i^j
        \end{align*} 
        which gives us $\reln{w_i^j}{iwo}{w_i^j}$.
        Since implied write orders are only between distinct writes, the property is proven.  
    \end{proof}
        
%-------------------------------------------------------------------------------------------------------------------------------------

    \begin{property}
        Implied write orders between two symmetric threads are reversed when they are swapped.
    \end{property}
        
    \begin{proof}
        Considering first sequential composition, i.e. $w_i^j;rf;po;w_k^j$, expanding gives us the following binary relations involved:
        \begin{align}
            \reln{w_i^j}{rf}{r_k} \\
            \reln{r_k}{po}{w_k^j}
        \end{align}
        Swapping thread identities involves swapping the indices $i$ and $k$ for each event, thus giving us 
        \begin{align}
            \reln{w_k^j}{rf}{r_i} \\
            \reln{r_i}{po}{w_i^j}
        \end{align}
        Through sequential composition of the above, we get $w_k^j;rf;po;w_i^j$, which by definition is $\reln{w_k^j}{iwo}{w_i^j}$.

        The argument is symmetric for the second sequential composition. 
    \end{proof}
        
%-------------------------------------------------------------------------------------------------------------------------------------

    \begin{property}
        There are at most two implied write orders between writes of two threads.
    \end{property}
        
    \begin{proof}
        Consider two threads $T_i$ and $T_k$. Suppose we have one implied write order between one of their writes, i.e. 
        \begin{align}
            \reln{w_i^j}{iwo}{w_k^j}.    
        \end{align}
        Expanding as per the first sequential composition gives us 
        \begin{align}
            w_i^j;po;rf^{-1};w_k^j
        \end{align}
        which also indicates a $\stck{_{rf}}$ with $T_i$'s read and that the writes involved in the composition are above the respective reads.

        Now suppose we have an implied write order between another set of writes, i.e.
        \begin{align}
            \reln{w_i^l}{iwo}{w_k^l}.    
        \end{align}
        Expanding as per the first sequential composition of implied write order is not possible as $\stck{_{rf}}$ is functional. Hence, using the second we have
        \begin{align}
            w_i^j;rf;po;w_k^j 
        \end{align}  
        which also indicates a $\stck{_{rf}}$ with $T_i$'s read that the writes involved in the composition are below the respective reads.

        Since both reads are now involved in a $\stck{_{rf}}$ relation, and since this relation is functional, we cannot have any more implied write orders between $T_i$ and $T_j$, thus verifying our property. 

        \critic{blue}{Better written in contrast to previous argument.}
        \critic{red}{However, is it necessary to show by contradiction?}
    \end{proof}
        
%--------------------------------------------------------------------------------------------------------------------------------------

    \begin{property}
        Implied write ordes between two threads are either all compliant with $stck{_{smo}}$ or they are all not
    \end{property}

    \begin{proof}
        
        If each read reads from its own write, we have no implied write order established, thus maintaining the property. 

        For cases where implied write orders are established, without loss of generality, let us consider one between writes above the read are compliant with $stck{_{smo}}$:
        \begin{align}
            \reln{w_i^j}{iwo}{w_k^j} \ \wedge \ \reln{w_i^j}{smo}{w_k^j}
        \end{align}
        The other set of implied write order, if established can only be between writes below the read. Suppose we have such an order but not compliant with $stck{_{smo}}$:
        \begin{align}
            \reln{w_k^l}{iwo}{w_i^l} \ \wedge \ \reln{w_i^l}{smo}{w_k^l}
        \end{align}
        Upon expanding using the second sequential composition (because writes are below the read), we get
        \begin{align}
            w_k^l;rf;po;w_i^l 
        \end{align}
        But this implies another $\stck{_{rf}}$ relation with $T_i$'s read, which violates the functional property of it. Hence we can only have an implied write order compliant with $stck{_{smo}}$.
        \begin{align}
            \reln{w_i^l}{iwo}{w_k^l} \ \wedge \ \reln{w_i^l}{smo}{w_k^l}
        \end{align}

        \critic{red}{Not sure if we need to show that the compliant relation also holds as it only brings an rf realtion with $T_k$'s read, which wanst established before.}

        The opposite case would make both the implied write orders requiring to not be compliant, thus by symmetry completing our proof. 
    \end{proof}

%-------------------------------------------------------------------------------------------------------------------------------------- 
        
    \begin{property}
        Implied write orders are acyclic
    \end{property}

    \begin{proof}
        %By Contradiction 
        Suppose a cycle exists. Then without loss of generality, we can consider the cycle composed of 3 writes.
        \begin{align}
            \reln{w_i^j}{iwo}{w_k^j} \ \wedge \reln{w_k^j}{iwo}{w_l^j} \ \wedge \reln{w_l^j}{iwo}{w_i^j}  
        \end{align}
        
        If these writes are above the read, then we have the following relations that result in the above cycle.
        \begin{align}
            \reln{w_i^j}{po}{r_i} \ \wedge \ \reln{r_i}{rf^{-1}}{w_k^j} \\
            \reln{w_k^j}{po}{r_k} \ \wedge \ \reln{r_k}{rf^{-1}}{w_l^j} \\
            \reln{w_l^j}{po}{r_l} \ \wedge \ \reln{r_l}{rf^{-1}}{w_i^j} 
        \end{align}
        The above relations form a cycle thus violating $po \cup rf^{-1} \ \text{acyclic}$ rule for coherence and hence violating coherence.
        
        If these writes are below the read, then we have the following relations that result in the above cycle.
        \begin{align}
            \reln{w_i^j}{po}{r_i} \ \wedge \ \reln{r_i}{rf}{w_k^j} \\
            \reln{w_k^j}{po}{r_k} \ \wedge \ \reln{r_k}{rf}{w_l^j} \\
            \reln{w_l^j}{po}{r_l} \ \wedge \ \reln{r_l}{rf}{w_i^j} 
        \end{align}
        The above relations form a cycle thus violating $po \cup rf \ \text{acyclic}$ rule for coherence and hence violating coherence.

        Because both cases violate coherence, we conclude that $\stck{_{iwo}}$ must be acylic. 

        \critic{blue}{A bit better written compared to the previous proofs}
    \end{proof}

%---------------------------------------------------------------------------------------------------------------------------------------

    \critic{red}{Need to also prove the properties of incmoing or outgoing write orders.}

    \subsubsection{Soundness}

        To prove that our rules our sound, we show that for every implied write order such that $smo;iwo$ is reflexive, we can swap the corresponding thread identities to reverse the write order between them, thus respectiing our irreflexivity constraint. However, we must ensure that once write orders are "fixed" in this fashion, they remain fixed, i.e. the relation cannot appear again to be wrong. Once we show this, we also need to show that this holds in general given multiple sets of equal writes. 
        
        \paragraph{Part1}
            For a given set of equal writes, once an implied write order is fixed, it remains fixed. 
            (A more formal statement required)
        \begin{proof}
            %Proof by contradiction 
            Suppose we have one implied write order which is correct / fixed between writes $w_i^j$ and $w_k^j$.
            \begin{align}
                \reln{w_i^j}{iwo}{w_k^j}
            \end{align}
            Without loss of generality, suppose these writes are above the read. We then divide our concern into two parts, one with implied write orders with $w_i^j$ which are wrong and the second with those of $w_k^j$.

            Case1: 
            Because $w_i^j$ is a write above read and $\reln{w_i^j}{iwo}{w_k^j}$, other implied write orders with $w_i^j$ can be of the form:
            \begin{align}
                \reln{w_m^j}{iwo}{w_i^j}
            \end{align}
            If this implied order is wrong, swapping thread identities will  give us the following relations 
            \begin{align}
                \reln{w_i^j}{iwo}{w_m^j} \ \wedge \ \reln{w_m^j}{iwo}{w_k^j}  
            \end{align}  
            Here, there is no implied write order relation between $w_i^j$ and $w_k^j$, hence we can consider them as remaining fixed. 
            If the implied order between $w_m^j$ and $w_k^j$ is wrong, swapping it will result in the following relations
            \begin{align}
                \reln{w_i^j}{iwo}{w_k^j} \ \wedge \ \reln{w_k^j}{iwo}{w_m^j}  
            \end{align}  
            Here, the implied write order between $w_i^j$ and $w_k^j$ remains the same. Thus, for this case, we can conclude that the relation remains "fixed".

            Case2: 
            Other implied write orders with $w_k^j$ can be of the forms:
            \begin{align}
                \reln{w_m^j}{iwo}{w_k^j} \\ 
                \reln{w_k^j}{iwo}{w_m^j}
            \end{align} 
            The first form is symmetric to our Case1, hence we only consider the second form. 

            If this implied write order is wrong, swapping thread identities will give us the following relations. 
            \begin{align}
                \reln{w_i^j}{iwo}{w_m^j} \ \wedge \ \reln{w_k^j}{iwo}{w_m^j}
            \end{align}
            Here, there is no implied write order relation between $w_i^j$ and $w_k^j$, hence we can consider them as remaining fixed. 

            But if the implied write order between $w_m^j$ and $w_i^j$ is wrong, swapping them will give us 
            \begin{align}
                \reln{w_m^j}{iwo}{w_i^j} \ \wedge \ \reln{w_k^j}{iwo}{w_i^j}  
            \end{align}
            thus making our claim invalid. To show that this state is not possible, note firstly that from the initial configuration, we can infer if $\reln{w_m^j}{iwo}{w_k^j}$ is wrong:
            \begin{align}
                \reln{w_i^j}{smo}{w_k^j} \ \wedge \ \reln{w_k^j}{smo}{w_m^j}
            \end{align} 
            Because $smo$ is a total order w.r.t. one set of writes, we have by transitivity. 
            \begin{align}
                \reln{w_i^j}{smo}{w_m^j}
            \end{align}
            After swapping threads $T_k$ and $T_m$, we get $\reln{w_i^j}{iwo}{w_m^j}$ which respects the irreflexivity constraint $smo;iwo \ \text{scyclic}$. Hence, this implied write order is not wrong. Thus we cannot have the case which results in $\reln{w_k^j}{iwo}{w_i^j}$.  

            Thus, for a given set of equal writes, once an implied write order is fixed, it remains fixed.
        \end{proof}

        \paragraph{Part2}
        For a given set of writes, any new implied write order introduced is new and, if wrong, it can be fixed and will remain fixed. 
        (A more formal statement required)
        \begin{proof}
            Suppose, for a given set of equal writes, say of the form $w^j$ all the implied write orders are fixed. Without loss of generality, let us consider them to be wrties above the read. We consider two threads $T_i$ and $T_k$, between which an implied write order is wrong. Let those writes be $w_i^l$ and $w_k^l$. 

            Once again, without loss of generality, let us consider the symmetric memory order between writes of $T_i$ and $T_k$ to be of the form $\reln{w_i}{smo}{w_k}$. Thus we have 
            \begin{align}
                \reln{w_i^j}{smo}{w_k^j} \ \wedge \ \reln{w_i^l}{smo}{w_k^l}
            \end{align}

            Because we assume write order is wrong between $w_i^l$ and $w_k^l$, by property x (forgot the number), and by our assumptions that implied write orders among $w^j$ are fixed, we have 
            \begin{align}
                \reln{w_k^l}{iwo}{w_i^l}
            \end{align} 
            There cannot be an implied write order between $w_i^j$ and $w_k^j$, as if there were, then it would be the fixed one, and by property y (forgot number), that between $w_i^l$ and $w_k^l$ would also be the fixed one. 

            \textbf{Case1: $w^l$ is below the read.}
            
                Part1: There exists an implied write order between another write $w^j$ and $w_i^j$. 

                    Because $w^l$ is below the read and $\reln{w_k^l}{iwo}{w_i^l}$, $w_i^j$ can only be involved in implied write orders of the form 
                    \begin{align}
                        \reln{w^j}{iwo}{w_i^j}
                    \end{align}
                    From this and the fact that impleid write orders of $w^j$ are fixed already, we have
                    \begin{align}
                        \reln{w^j}{smo}{w_i^j} 
                    \end{align} 
                    By transitivity, we then have $\reln{w^j}{smo}{w_k^j}$.

                    If we swap $T_i$ and $T_k$ to fix $\reln{w_k^l}{iwo}{w_i^j}$, we have the following new relations.
                    \begin{align}
                        \reln{w_i^l}{iwo}{w_k^j} \\
                        \reln{w^j}{iwo}{w_k^j}
                    \end{align}
                    Both these relations repsect our irreflexivity constraint. Hence, maintaining that new implied write order relations with $w^j$ are not wrong.
                    
                Part2: There exists an implied write order between another write $w^j$ and $w_k^j$. 
                    
                    $w_k^j$ can be involved in implied write orders of the form
                    \begin{align}
                        \reln{w_k^j}{iwo}{w^j} \\
                        \reln{w^j}{iwo}{w_k^j} 
                    \end{align}
                    The first is symmetric to Part1, hence we only consider the second form. 

                    On swapping $T_i$ and $T_k$, we get the following new relations:
                    \begin{align}
                        \reln{w_i^l}{iwo}{w_k^j} \\
                        \reln{w^j}{iwo}{w_i^j}
                    \end{align}

                    The second form may not be compliant to the irreflexivity condition, hence can be a new implied relation which is wrong. To show that this is new and could have not occurred before while swapping threads to fix implied write orders of $w^j$, consider the original configuration we have 
                    \begin{align}
                        \reln{w_i^j}{smo}{w_k^j} \ \wedge \ \reln{w^j}{iwo}{w_k^j}
                    \end{align}
                    For $\reln{w^j}{iwo}{w_k^j}$ to have been there before, there must be a write $x$, say, such that 
                    \begin{align}
                        \reln{x}{iwo}{w_i^j} \ \wedge \ \reln{x}{iwo}{w^j}
                    \end{align}
                    So that while swapping to reverse relation between $x$ and $w^j$ or $x$ and $w_i^j$, we can get our new relation. But such an $x$ cannot exist, as by property z (forgot number), writes above read, can only have one relation of the form $\reln{x}{iwo}{y}$ where $y$ is another equal write. Hence, such a relation could have not been there before, and hence was not fixed. 

                    Now that this new relation exists, fixing it, will keep it remain fixed, by our first part of proof (label them please). 
                    
            
            \textbf{Case2: $w^l$ is above the read.}
                
                Part1: There exists an implied write order between another write $w^j$ and $w_i^j$.

                Since $T_i$'s read is still free, $w_i^j$ can have implied write orders with $w^j$ of the form
                \begin{align}
                    \reln{w_i^j}{iwo}{w^j} \\
                    \reln{w_i^j}{iwo}{w^j} 
                \end{align}

                \critic{blue}{This case is symmetric to the above, so we will write it later.}

                For the second form, while we swap threads $t_i$ and $T_k$, we get the following new relations (correct his)
                \begin{align}
                    \reln{w_i^l}{iwo}{w_k^j} \\
                    \reln{w_k^j}{iwo}{w^j}
                \end{align}
                The second relation may not be compliant to the irreflexivity condition, hence can be a new implied relation which is wrong. To show that this is new and could have not occurred before while swapping threads to fix implied write orders of $w^j$, consider the original configuration we have 
                \begin{align}
                    \reln{w_i^j}{smo}{w_k^j} \ \wedge \ \reln{w_i^j}{iwo}{w^j}
                \end{align}

                For $\reln{w^j}{iwo}{w_k^j}$ to have been there before, we need an event $x$ to connect, hence by possible implied write orders one can have with $w^j$ and $w_k^j$, we need an $x$ such that one of the conditions below hold
                \begin{align}
                    \reln{w^j}{iwo}{x} \ \wedge \ \reln{x}{iwo}{w_k^j} \\ 
                    \reln{x}{iwo}{w^j} \ \wedge \ \reln{x}{iwo}{w_k^j}
                \end{align}

                The first condition violates coherence ($po \cup rf_{-1}$ acylcic), while the second condition cannot hold as such an event $x$ cannot exist. 
                
                Hence, such a relation could have not been there before, and hence was not fixed. 

                Now that this new relation exists, fixing it, will keep it remain fixed, by our first part of proof (label them please).

                \critic{red}{Write the above argument better please.}

        \end{proof}

        

   
\documentclass{article}

\usepackage{geometry}
\usepackage{amsmath}
\usepackage{graphicx}
\usepackage{float}
\usepackage{amsthm}
\usepackage[dvipsnames]{xcolor}

%Short form to use stack_relative
\newcommand{\stck}{\stackrel{\longrightarrow}}

%A different version of the above 
\newcommand{\stckdet}[1]{\stackrel{{#1}}}

%We write a lot of relations between two events using the orderings, so a short form to use that
\newcommand{\reln}[3]{#1\stck{_{#2}}#3}

%We use a more detailed version of the above relation to indicate direct / indirect relations 

\newcommand{\reldet}[4]{#1\stckdet{_{#2}}{{\stck{_{#3}}}}#4}

%We also will introduce a short form to write an event belongs to some set
\newcommand{\event}[2]{#1\!\in\!#2}

%To make events and their type more close to each other
\newcommand{\typ}[1]{\textit{#1}}
\newcommand{\et}[2]{#1\!:\!\typ{#2}}

%Short form to write color text
\newcommand{\critic}[2]{\textcolor{#1}{\footnotesize #2}}

%Some preliminary latex commands to format writing theorems 

\newtheorem{lemma}{Lemma}

\newtheorem{theorem}{Theorem}[lemma]

\newtheorem{corollary}{Corollary}[theorem]

\newtheorem{definition}{Definition}

\newtheorem{property}{Property}[section]

\newtheorem{case}{Case}[lemma]


\begin{document}

    
        %\section{Objective}

        This whole document's purpose is to gain clarity over certain ideas that have been formulated in my head. 
        Usage of figures in this document will be heavy just to point out an intuition. 
        This document is also the first one written in latex in this respect. 
        The author of this article has taken a lot of effort just to start formulating ideas in latex format. 
        So if there are certain typos or errors in text, please do not mind and kindly report it to the author. 

    \section{Topic : Symmetry Reduction}

        Model checking is heavily relied upon as a testing tool for programs. 
        Given that programs are large model checking suffers from its constant problem of scalability. 
        Add to the above the introduction of concurrent programs, scalability becomes more of an issue. 
        
        However, to optimize model checking of concurrency, certain algorithms rely on reusing information from one execution. 
        In that sense, a certain dynamic programming approach is employed. 
        This dynamic approach is done to avoid certain redundant explorations, as well as to quickly discard further exploration of programs that violate the concurrent semantics.
        In the ones that I have read, which is called in literature as stateless model checking, storing the information of the writes each read operation has visited in some execution explored before is used to ensure optimal model checking. 
        
        But here the meaning of redundant is that of executions that the model checker has already explored. 
        There exists another kind of redundancy which is the fact that two different executions result in the same outcome. 
        This kind of redundancy relies more upon the semantic equivalence of two executions, rather than just syntactic. 

        One such instance of such redundancy is the equivalence(not equality) of exeuction graphs.
        The equivalence is sort of homomorphic as well as isomorphic. 
        Such kind of redundancy appears in concurrent programs due to different components of the program tasked to do mostly the same thing. (take for example, a client-server system)
        Such equivalent executions can be said to be symmetric to each other. 
        This means that one needs to explore just one of these executions to ensure the other symmetric ones are also explored.


        %\subsection{Graph and its Equivalence}

    \paragraph{Execution Graph}
    
        \begin{itemize}
            \item Has all the ordering relations defined by the axiomatic model (eg: happens-before, program-order)
            \item Each execution graph is unique
        \end{itemize}
    
    
    \paragraph{Equivalence}
        \begin{center}
            \emph{\textit{
                Two execution graphs E1 and E2 are symmetric / equivalent to each other, if we can swap thread identities of one of them to form the other. 
            }}  
        \end{center}

        In terms of ordering relations, swapping thread identities means: 
        \begin{itemize}
            \item Any reads-from relation established among events of the threads being swapped are also swapped.
            \item Reads-from relation from read of a thread being swapped to an outside write is changed such that the read is that of a swapped thread having the same relation with the write.   
        \end{itemize}

        \critic{blue}{There might be other derived relations, eg: happens-before, from-reads, etc. For now we only keep it restricted to reads-from relations.}

        In short, if we tag all the thread events by some number, simply changing the numbers by their swapped thread identities will ensure the above two. 

        \critic{blue}{Note that it could be the case that an execution has two threads to be of same code, but it may not reflect what it is in the original program. The equivalence is on the assumption that both threads do have equal code in both those  executions.}

    \paragraph{Example of swapping thread identities?}
        
        %Show an example here
        \begin{figure}[H]
            \centering
            \includegraphics[scale=0.7]{Equivalence_Example.pdf}
            \caption{Example of two equivalent execution graphs}
        \end{figure}

        The two executions which are symmetric have the following reads-write relations. 
        \begin{align*}
            b1:x \ \wedge \ b2:a1 \\
            b1:a2 \ \wedge \ b2:x 
        \end{align*}

        Notice from the above figure that swapping thread identities of T1 and T2 implies jsut swapping $\{b1, b2\}$ and $\{a1, a2\}$.
        
        \critic{blue}{It took quite some time to have a clear picture of what swapping thread identities implied for us when looking at symmetric executions.}




        
        %\subsection{The Intuition}
    
    \paragraph{Order Between Symmetric Threads}
        \begin{itemize}
            \item Symmetric executions can be obtained by swapping thread identities.
            \item Hence without loss of generality, we can assume a single total order of all threads which have equal code.
            \item In this sense, each set of equal threads will have a total order assigned to them. 
        \end{itemize}

    \paragraph{Order between Events - Writes and Reads}
        \begin{itemize}
            \item Whenever a read-write relation implies an order between two writes in a symmetric thread, swapping their thread identities will give us a symmetric execution with the order between writes reversed.
            \item Hence maintaining a symmetric order between writes of threads with equal code might be useful. 
            \item However, do we need a fixed order between all writes or just among writes which are equal in each thread? 
            \item Order between reads may not be necessary as they are not implied due to any other ordering-relation.  
        \end{itemize}

    \critic{red}{Whether order between reads is important is something I have a counter example to show that it isn't. But  I am not able to understand why it isn't important.}

    \paragraph{Miscellaneous Points}

        About ordering between writes through reads. 
        \begin{itemize}
            \item Each thread's read can read from it's own write program ordered before it.
            \item Each thread's read can read from an outer write which implies an ordering between the writes in it's own thread and the external one from which it reads from. 
        \end{itemize}

        \begin{figure}[H]
            \centering
            \includegraphics[scale=0.7]{WriteOrderImplied.pdf}
            \caption{Example showing that write orders(mo) implied due to the read-write(rf) relation.}
        \end{figure}




























   

        %\subsection{From Intuition to Examples}

    
    \emph{Any reads-from relation between a read (r) and a write (w) will be written as \textbf{r:w}} 


    \paragraph{Examples to show Write orders important}

        The following two examples show why having a symmetric order between same writes of each equal thread is important. 
        \begin{figure}[H]
            \centering 
            \includegraphics[scale=0.7]{Example1(2+wr).pdf}
            \caption{2+wr}
        \end{figure}

        The above example has the following possibilities of read-write relations that result due to an execution. Note that $b1:a2 , b2:a1$ is not possible due to coherence being our assumption. 
        \begin{align*}
            b1:a1 , b2:a1 \\ 
            b1:a1 , b2:a2 \\
            b1:a2 , b2:a2
        \end{align*}

        Observations:
        \begin{itemize}
            \item $b1:a1, b2:a1$ implies an ordering on the writes such that $a2$ occurs before $a1$.
            \item $b1:a2, b2:a2$ implies an ordering on the writes such that $a1$ occurs before $a2$.
            \item The above two executions are symmetric to each other as we can swap threads of one to get the other. 
            \item We need to fix an ordering between the equal writes of symmetric threads. 
        \end{itemize}
    
        \begin{figure}[H]
            \centering 
            \includegraphics[scale=0.7]{Example2(2+wrw).pdf}
            \caption{2+wrw}
        \end{figure}
        
        The above example is an extension of the previous, by adding two writes following the reads to the same memory. Note that the outcome $b1:c2, b2:c1$ is not considered due to lack of Load-Buffering (I still think its Coherence). The following are the possible executions in terms of read-write relations. 

        \begin{align*} 
            b1:a1 , b2:a2 \\
            b1:a1 , b2:c1 \\ 
            b1:a2 , b2:a2 \\
            b1:a2 , b2:c1 \\
        \end{align*}

        Observations:
        \begin{itemize}
            \item $b1:c2, b2:a2$ is not counted as it implies that $c2$ is before $c1$, whose reverse is already covered by $b1:a1 , b2:c1$
            \item Same argument for why the case $b1:c2, b2:a1$ is not considered.
            \item A slight hint is given that ordering between reads also matters. (the case $b1:c2, b2:a2$ also implies $b2$ occurs before $b1$.) 
        \end{itemize}

    \paragraph{Examples to show Read Orders are(or may be) not important}

        \begin{figure}[H]
            \centering 
            \includegraphics[scale=0.7]{Example3(3+wr).pdf}
            \caption{3+wr}
        \end{figure}


        The above example is extending the first example with another symmetric thread. It has the following executions covered using our insight from the previous examples. 
        \begin{align*}
            b1:a1 , b2:a2 , b3:a3 \\
            b1:a1 , b2:a3 , b3:a3 \\
            b1:a2 , b2:a2 , b3:a3 \\
            b1:a2 , b2:a3 , b3:a3 \\
            b1:a3 , b2:a2 , b3:a3 \\
            b1:a3 , b2:a3 , b3:a3
        \end{align*}

        Observation:
        \begin{itemize}
            \item $b1:a1 , b2:a3 , b3:a3$ and $b1:a2 , b2:a2 , b3:a3$ appear to be symmetric, but considering the total order between equal writes also as part of execution, they are not symmetric, rather they are unique. 
            \item  $b1:a3 , b2:a2 , b3:a3$ is a case we would not have covered, if we enforced an order between reads, as it would then mean that $b2$ can only read from $a3$. 
            \item Enforcing read orders prevent the above case. The above case is valid because $b2:a2$ implies no ordering between writes, hence the order between $a1$ and $a2$ can be left to us. 
            \item \textcolor{red}{However, the above case is symmetric to $b1:a2 , b2:a2 , b3:a3$; swap identities of thread $T2$ and $T3$}
            \item This strengthens our intuition of equivalent executions that can be obtained by swapping thread identities, due to reversing implied orders between writes. 
        \end{itemize}

        \critic{red}{Not enforcing an order between reads, may have issues about completeness, meaning we might cover a few redundant explorations.}

        \begin{figure}[H]
            \centering
            \includegraphics[scale=0.7]{Example4(3+wrw).pdf}
            \caption{3+wrw}
        \end{figure}

        The above example is extending the previous with additional writes to each thread. The possible executions are many, hence not all of them will be elicited, but ones of focus: 
        
        \begin{align*}
            b1:a1 , b2:a3 , b3:c1 \\
            b1:a3 , b2:a2 , b3:c2 \\
            b1:a1 , b2:a2 , b3:c1 \\
            b1:a2 , b2:a2 , b3:c1 \\
        \end{align*}

        Observation:
        \begin{itemize}
            \item $b1:a1 , b2:a3 , b3:c1$ would not be explored if we consider $b1:a1 , b2:a3 , b3:a3$ to be symmetric to $b1:a2 , b2:a2 , b3:a3$. Considering them symmetric would make us assume that when $b1:a1$, then $b2$ cannot read from any outside write other than that of $T1$. 
            \item $b1:a3 , b2:a2 , b3:c2$ would not be explored if we ordered the reads too. However, it is also the case that $b1:a1 , b2:a3 , b3:c1$ is symmetric to this (just swap thread identities of $T1$ and $T2$). Further investigation is needed. This tells our approach may not be complete. 
            \item \textcolor{red}{$b1:a1 , b2:a2 , b3:c1$ is symmetric to $b1:a1 , b2:a2 , b3:c2$ , but it does not violate our requirement of implied write orders.}
            \item \textcolor{red}{$b1:a2 , b2:a2 , b3:c1$ is symmetric to $b1:a3, b3:a3, b2:c1$, but it does not violate our requirement of implied write orders.} 
            \item Conclusion from this example is that implied write orders is not complete. 
        \end{itemize}

        
    \paragraph{Examples to show affects of external non-symmetric writes}
        %Now consider extending the first example with another non symmetric thread 2+wr+w
        \begin{figure}[H]
            \centering 
            \includegraphics[scale=0.7]{Example5(2+wr+w).pdf}
            \caption{2+wr+w}
        \end{figure}

        The above example is the first program extended with another non-symmetric thread having a write to same memory. We elicit the following possible executions, catering to our rules for implied write orders: 
        \begin{align*}
            b1:a1 , b2:a2 \\
            b1:a1 , b2:x \\
            b1:a2 , b2:a2 \\
            b1:a2 , b2:x \\
            b1:x , b2:x \\
            b1:x , b2:a2
        \end{align*}

        Observations: 
        \begin{itemize}
            \item There is no restriction on order with the write $x$ and $a1, a2$.
            \item $b1:x , b2:a2$ is symmetric to $b1:a1 , b2:x$. But both of these explored do not violate implied write orders. 
            \item $b1:x , b2:a2$ can correspond to an order of writes being $a1->x->a2$ still respecting the order of writes that must hold. Whereas, $b1:a1 , b2:x$ can correspond to the order of writes being $a1->a2->x$ also respecting the order of writes that must hold. 
        \end{itemize}

        \begin{figure}
            \centering
            \includegraphics[scale=0.7]{Example6(2+wr+2+w).pdf}
        \end{figure}

       
        The above example is with two symmetric sets of threads. We have the following executions possible keeping our implied writes rule intact. 

        \begin{align*}
            b1:a1 , b2:a2 \\
            b1:a1 , b2:x1 \\
            b1:a1 , b2:x2 \\
            b1:a2 , b2:a2 \\
            b1:a2 , b2:x1 \\
            b1:a2 , b2:x2 \\
            b1:x1 , b2:a2 \\
            b1:x1 , b2:x2 \\ 
            b1:x2 , b2:a2 \\
            b1:x2 , b2:x2 \\
            b1:x2 , b2:x1
        \end{align*}

        Observation:
        \begin{itemize}
            \item \textcolor{red}{$b1:x1 , b2:x2$ is symmetric to $b1:x2 , b2:x1$, yet these two do not violate the implied writes.} rules. 
        \end{itemize}
        
        Notice that we do not have the case $b1:x2 , b2:x1$, as it implicitly violates the ordering constraint that $b1$ occurs before $b2$.

        (Still ongoing, examples with operations to multiple memory segments need to be analyzed)

        %Lastly consider the case 3+wrw+w

        %Keep this section open to more interesting examples

        \subsection{From Examples to Precise Rules}

    \paragraph{Notations}
        \begin{itemize}
            \item $T_i$ denotes thread number $i$.
            \item $T_i \equiv T_j$ means both threads have same code.
            \item $w_i^j$ is the $j^{th}$ event in thread $i$ which is a write.
            \item $r_i^j$ is the $j^{th}$ event in thread $i$ which is a read. 
        \end{itemize}

    
    \paragraph{A few definitions for our use}

    \begin{definition}{Program Order(\emph{po})}
        Total order between events in the same thread. Respects the execution order between events in the same thread. 
    \end{definition}

    \begin{definition}{Symmetric Memory Order (\emph{smo})}
        \label{SymMemO}
        A strict partal order between writes in a set of symmetric threads. Consider a set of symmetric threads $T_1 \equiv T_2 \equiv ... \equiv T_n$. Each of these threads have exactly one read event, and multiple write events, all to the same memory, say $x$. 

        Then each write in the above threads are involved in a symmetric order, such that. 
        \begin{align*}
            \forall i \in [0, n-1] \ . \ \reln{w_i^j}{smo}{w_{i+1}^j}
        \end{align*}
        Where j denotes the $j^{th}$ event in any of the threads, which is a write.
    \end{definition}

    \critic{blue}{Perhaps should put examples for the above defintion.}

    \begin{definition}{Reads-From (\emph{rf})}
        \label{ReadF}
        Binary relation that links a read to a write from which its value comes. Note that for our purpose, this relation is functional.
        For example, if a read $r_i^j$ gets its read value from write $w_k^l$, then we have the relation. 
        \begin{align*}
            \reln{w_k^l}{rf}{r_i^j}
        \end{align*}
    \end{definition}

    \paragraph{Main Rule}
        Using the above setup, our intention is to explore lesser execution graphs leveraging the symmetry that can result due to swapping of thread identities. For this, we enforce a restriction on possible $\stck{_{rf}}$ relations that are to be considered \emph{valid}. A valid $\stck{_{rf}}$ relation is one that respects the following \textbf{irreflexivity constraints}. 
        \begin{align*}
            smo;rf;po \\
            smo;po;rf^{-1} 
        \end{align*}

        \critic{blue}{Perhaps consider labelling this rule}

        \critic{blue}{We can add here more as we go about to prove completeness.}

        \critic{red}{Recall examples to show how our analysis through examples satisfy the above irreflexivity constraint.}

        



        \subsection{Soundness of the rules above}

    To prove soundness, we first define the following: 

    \begin{definition}{Implied Write Order(\emph{iwo})}
        \label{ImpliedW}
        Binary relation between any two \emph{distinct} writes, derived through the following two sequential conposition:  
        \begin{align*}
            w_i^j;po;rf^{-1};w_k^j \\
            w_i^j;rf;po;w_k^j 
        \end{align*}
    \end{definition}

%---------------------------------------------------------------------------------------------------------------------------------    

    \begin{property}{Simplified irreflexivity rule}
        \label{prop1}
        The irreflexivity constraint rule is equivalent to the following irreflexivity condition 
        \begin{align}
            smo;iwo
        \end{align}
    \end{property}

    \begin{proof}
        Expanding for implied write order as per the definition, gives us the following two sequential compositions. 
        \begin{align}
            smo;w_i^j;po;rf^{-1};w_k^j \\
            smo;w_i^j;rf;po;w_k^j 
        \end{align}
        If the above relations must be irreflexive, then so should the following: 
        \begin{align}
            w_k^j;smo;w_i^j;po;rf^{-1} \\
            w_k^j;smo;w_i^j;rf;po     
        \end{align}
        By \ref{SymMemO}, the above can be simplified to
        \begin{align}
            smo;po;rf^{-1} \\
            smo;rf;po 
        \end{align}
        Thus, proving our property. 
    \end{proof}

%-----------------------------------------------------------------------------------------------------------------------------------

    \begin{property}
        \label{prop2}
        No write order is implied when a read reads from its own thread's write
    \end{property}

    \begin{proof}
        If the read is from its own thread's write, then we can infer that $i=k$ in both the sequential compositions. Hence  
        \begin{align*}
            w_i^j;rf;po;w_i^j \\
            w_i^j;po;rf^{-1};w_i^j
        \end{align*} 
        which gives us $\reln{w_i^j}{iwo}{w_i^j}$.
        Since implied write orders are only between distinct writes, the property is proven.  
    \end{proof}

    \critic{red}{Perhaps define implied write order as being irreflexive}
        
%-------------------------------------------------------------------------------------------------------------------------------------

    \begin{property}
        \label{prop3}
        Implied write orders between two symmetric threads are reversed when they are swapped.
    \end{property}
        
    \begin{proof}
        Considering first sequential composition, i.e. $w_i^j;rf;po;w_k^j$, expanding gives us the following binary relations involved:
        \begin{align}
            \reln{w_i^j}{rf}{r_k} \\
            \reln{r_k}{po}{w_k^j}
        \end{align}
        Swapping thread identities involves swapping the indices $i$ and $k$ for each event, thus giving us 
        \begin{align}
            \reln{w_k^j}{rf}{r_i} \\
            \reln{r_i}{po}{w_i^j}
        \end{align}
        Through sequential composition of the above, we get $w_k^j;rf;po;w_i^j$, which by definition is $\reln{w_k^j}{iwo}{w_i^j}$.

        For the second sequential composition, i.e. $w_i^j;po;rf^{-1};w_k^j$, expanding gives us the following binary relations involved:
        \begin{align}
            \reln{w_i^j}{po}{r_i} \\
            \reln{r_i}{rf^{-1}}{w_k^j}
        \end{align}
        Swapping thread identities $T_i$ and $T_k$ gives us the following relations 
        \begin{align}
            \reln{w_k^j}{po}{r_k} \\
            \reln{r_k}{rf^{-1}}{w_i^j}
        \end{align}
        Whose sequential compostiion gives us $w_k^j;po;rf^{-1};w_i^j$, which by defintion is $\reln{w_k^j}{iwo}{w_i^j}$.
    \end{proof}
        
%-------------------------------------------------------------------------------------------------------------------------------------

    \begin{property}
        \label{prop4}
        There are at most two implied write orders between writes of two threads, with one between writes above the read event and one below. 
    \end{property}
        
    \begin{proof}
        Consider two threads $T_i$ and $T_k$. Suppose we have one implied write order between one of their writes, i.e. 
        \begin{align}
            \reln{w_i^j}{iwo}{w_k^j}.    
        \end{align}
        Expanding as per the first sequential composition gives us the following relations involved
        \begin{align}
            \reln{w_i^j}{po}{r_i} \ \wedge \ \reln{r_i}{rf^{-1}}{w_k^j} 
        \end{align}
        which indicates a $\stck{_{rf}}$ with $T_i$'s read and imlpied write order is between writes above the read. 

        Now suppose we have an implied write order between another set of writes, i.e.
        \begin{align}
            \reln{w_i^l}{iwo}{w_k^l}.    
        \end{align}
        By \ref{ReadF}, the first sequential composition of implied write order could not have formed the above relation; hence, expanding the second we have
        \begin{align}
            \reln{w_i^j}{rf}{r_k} \ \wedge \ \reln{r_k}{po}{w_k^j}
        \end{align}  
        which also indicates a $\stck{_{rf}}$ with $T_k$'s read that the writes involved in the composition are below the respective reads.

        Since both reads are now involved in a $\stck{_{rf}}$ relation, by \ref{ReadF}, we cannot have any more implied write orders between $T_i$ and $T_j$, thus verifying our property. 

        \critic{blue}{Could be made better}
    \end{proof}
        
%--------------------------------------------------------------------------------------------------------------------------------------

    \begin{property}
        \label{prop5}
        Implied write orders between two threads either all respect irreflexivity condition by \ref{prop1} or they all do not.  
    \end{property}

    \begin{proof}
        
        Consider our two threads $T_i$ and $T_k$. 

        If each of their read reads from its own write, we have no implied write order established, thus maintaining the property.
        
        If each read reads from another thread's (say $T_m$), write, we have no implied write order established between $T_i$ and $T_j$'s writes.. (I do not know how to complete the sentence.)

        For cases where implied write orders are established between wrties of $T_i$ and $T_j$, without loss of generality, let us consider one between writes above the read which respect \ref{prop1}; 
        \begin{align}
            \reln{w_i^j}{iwo}{w_k^j} \ \wedge \ \reln{w_i^j}{smo}{w_k^j}
        \end{align}
        The above implied write order satisfies $T_i$'s read by a reads-from relation with $w_k^j$. 

        The other set of implied write order, by \ref{prop5} can only be between writes below the read. Suppose we have such an order but not respecting \ref{prop1}:
        \begin{align}
            \reln{w_k^l}{iwo}{w_i^l} \ \wedge \ \reln{w_i^l}{smo}{w_k^l}
        \end{align}
        Upon expanding using the second sequential composition (because writes are below the read), we get
        \begin{align}
            w_k^l;rf;po;w_i^l 
        \end{align}
        This implies another $\stck{_{rf}}$ relation with $T_i$'s read. By \ref{ReadF} this should not be allowed. Hence we can only have an implied write order respecting \ref{prop1}.
        \begin{align}
            \reln{w_i^l}{iwo}{w_k^l} \ \wedge \ \reln{w_i^l}{smo}{w_k^l}
        \end{align}

        \critic{red}{The last argument seems incomplete. But perhaps compliant / non-compliant is a binary argument. So if one does not hold, the other should. There is no other option.}

        By symmetry, if the implied write order between writes above the read did not respect \ref{prop1}, then so would the writes below the read.  
    \end{proof}

%-------------------------------------------------------------------------------------------------------------------------------------- 
        
    \begin{property}
        \label{prop6}
        Implied write orders are acyclic
    \end{property}

    \begin{proof}
        %By Contradiction 
        Suppose a cycle exists. Then without loss of generality, we can consider the cycle composed of 3 writes.
        \begin{align}
            \reln{w_i^j}{iwo}{w_k^j} \ \wedge \reln{w_k^j}{iwo}{w_l^j} \ \wedge \reln{w_l^j}{iwo}{w_i^j}.  
        \end{align}
        
        By \ref{ImpliedW} we can have just two cases; the set of writes involved in cycle are either above the read event or below.

        If they are above the read, then we have the following relations that result in the cycle
        \begin{align}
            \reln{w_i^j}{po}{r_i} \ \wedge \ \reln{r_i}{rf^{-1}}{w_k^j} \\
            \reln{w_k^j}{po}{r_k} \ \wedge \ \reln{r_k}{rf^{-1}}{w_l^j} \\
            \reln{w_l^j}{po}{r_l} \ \wedge \ \reln{r_l}{rf^{-1}}{w_i^j}. 
        \end{align}
        The above relations violate $po \cup rf^{-1} \ \text{acyclic}$ rule, thus violating coherence.
        
        If the writes are below the read, then we have the following relations that result in the above cycle.
        \begin{align}
            \reln{w_i^j}{po}{r_i} \ \wedge \ \reln{r_i}{rf}{w_k^j} \\
            \reln{w_k^j}{po}{r_k} \ \wedge \ \reln{r_k}{rf}{w_l^j} \\
            \reln{w_l^j}{po}{r_l} \ \wedge \ \reln{r_l}{rf}{w_i^j} 
        \end{align}
        The above relations form a cycle thus violating $po \cup rf \ \text{acyclic}$ rule, thus violating coherence.

        Because both cases violate coherence, we conclude that $\stck{_{iwo}}$ is acylic. 
    \end{proof}

%---------------------------------------------------------------------------------------------------------------------------------------

    \begin{property}
        \label{prop7}
        Writes $w_i$ above read in each thread can only have one relation of the form $\reln{w_i}{iwo}{w}$ but can have several of the form $\reln{w}{iwo}{w_i}$. 
    \end{property}

    \begin{proof}
        Suppose a write $w_i^j$ above a read has a relation with $w_k^j$ such that $\reln{w_i^j}{iwo}{w_k^j}$. Expanding based on \ref{ImpliedW}, we get:
        \begin{align}
            \reln{w_i^j}{po}{r_i} \ \wedge \ \reln{r_i}{rf_{-1}}{w_k^j} \\ 
            \reln{w_i^j}{rf}{r_k} \ \wedge \ \reln{r_k}{po}{w_k^j}
        \end{align}
        Because $w_i^j$ is above the read, the first set of relations justify the implied write order and also implies that $T_i$'s read has been satisfied by write $w_k^j$. By \ref{ReadF} and our assumption of one read per thread, we cannot have any more implied write order relations of the form $\reln{w_i^j}{iwo}{w^j}$. 

        On the other hand, there can be many relations of the form $\reln{w_k^j}{iwo}{w_i^j}$ as $k$ can be identity of any thread, whose read gets satisfied by the argument above.
    \end{proof}

%---------------------------------------------------------------------------------------------------------------------------------------
   
    \begin{property}
        \label{prop8}
        Writes $w_i$ below read in each thread can only have one relation of the form $\reln{w}{iwo}{w_i}$ but can have several of the form $\reln{w_i}{iwo}{w}$. 
    \end{property}

    \begin{proof}
        Suppose a write $w_i^j$ below the read has a relation with $w_k^j$ such that $\reln{w_k^j}{iwo}{w_i^j}$. Expanding based on both forms of implied write order, we get:
        \begin{align}
            \reln{w_k^j}{po}{r_k} \ \wedge \ \reln{r_k}{rf_{-1}}{w_i^j} \\ 
            \reln{w_k^j}{rf}{r_i} \ \wedge \ \reln{r_i}{po}{w_i^j}
        \end{align}
        Because $w_i^j$ is below the read, the second set of relations justify the implied write order, implying that $T_i$'s read has been satisfied by write $w_k^j$. By \ref{ReadF} and our assumption of one read per thread, we cannot have any more implied write order relations of the form $\reln{w^j}{iwo}{w_i^j}$. 

        On the other hand, there can be many relations of the form $\reln{w_i^j}{iwo}{w_k^j}$ as $k$ can be identity of any thread, whose read gets satisfied by the argument above.  
    \end{proof}

%----------------------------------------------------------------------------------------------------------------------------------------

    \subsubsection{Soundness}

        To prove that our rules our sound, we show that for every implied write order such that $smo;iwo$ is reflexive, by \ref{prop3} we can reverse them by swapping thread identities, thus respectiing our irreflexivity constraint. 
        
        However, we must ensure that once write orders are "fixed" in this fashion, they remain fixed, i.e. the relation cannot appear again to be wrong. 
        For this, we first need to ensure that implied write orders do not form a cycle. By \ref{prop6}, this is indeed the case. 
        
        Given they are acyclic, we need to show that once fixed by \ref{prop3}, they remain fixed. 
        Lastly, we need to show that this holds in general given multiple sets of equal writes that needs fixing.  

%------------------------------------------------------------------------------------------------------------------------------------------

        \begin{lemma}
            For a given set of equal writes, once an implied write order is fixed, it remains fixed. 
            (A more formal statement required perhaps?)
        \end{lemma}
            
        \begin{proof}
            %Proof by contradiction 
            Suppose we have one implied write order which is correct / fixed between writes $w_i^j$ and $w_k^j$.
            \begin{align}
                \reln{w_i^j}{iwo}{w_k^j}
            \end{align}
            Without loss of generality, suppose these writes are above the read. We then divide our concern into two parts, one with implied write orders with $w_i^j$ which are wrong and the second with those of $w_k^j$.

            Case1: 
            Because \ref{prop7} and $\reln{w_i^j}{iwo}{w_k^j}$, other implied write orders with $w_i^j$ can be of the form:
            \begin{align}
                \reln{w_m^j}{iwo}{w_i^j}
            \end{align}
            If this implied order is wrong, swapping thread identities will  give us the following relations 
            \begin{align}
                \reln{w_i^j}{iwo}{w_m^j} \ \wedge \ \reln{w_m^j}{iwo}{w_k^j}  
            \end{align}  
            Here, there is no implied write order relation between $w_i^j$ and $w_k^j$, hence we can consider them as remaining fixed. 
            If the implied order between $w_m^j$ and $w_k^j$ is wrong, swapping it will result in the following relations
            \begin{align}
                \reln{w_i^j}{iwo}{w_k^j} \ \wedge \ \reln{w_k^j}{iwo}{w_m^j}  
            \end{align}  
            Here, the implied write order between $w_i^j$ and $w_k^j$ remains the same. Thus, for this case, we can conclude that the relation remains "fixed".

            Case2: 
            By \ref{prop7}, implied write orders with $w_k^j$ can be of the forms:
            \begin{align}
                \reln{w_k^j}{iwo}{w_m^j} \\
                \reln{w_m^j}{iwo}{w_k^j} 
            \end{align} 
            The first form is symmetric to Case1, hence we only consider the second form. 

            If this implied write order is wrong, swapping thread identities will give us the following relations. 
            \begin{align}
                \reln{w_i^j}{iwo}{w_m^j} \ \wedge \ \reln{w_k^j}{iwo}{w_m^j}
            \end{align}
            Here, there is no implied write order relation between $w_i^j$ and $w_k^j$, hence we can consider them as remaining fixed. 

            If the implied write order between $w_m^j$ and $w_i^j$ is wrong, swapping their threads will give us 
            \begin{align}
                \reln{w_m^j}{iwo}{w_i^j} \ \wedge \ \reln{w_k^j}{iwo}{w_i^j}  
            \end{align}
            thus making our claim invalid. To show that this state is not possible, note firstly that from the initial configuration, we can infer by \ref{SymMemO} and \ref{prop1}:
            \begin{align}
                \reln{w_i^j}{smo}{w_k^j} \ \wedge \ \reln{w_k^j}{smo}{w_m^j}
            \end{align} 
            Because $smo$ is a total order w.r.t. one set of writes, we have by transitivity. 
            \begin{align}
                \reln{w_i^j}{smo}{w_m^j}
            \end{align}
            After swapping threads $T_k$ and $T_m$, we get $\reln{w_i^j}{iwo}{w_m^j}$ which respects the irreflexivity constraint \ref{prop1}. Hence, this implied write order is not wrong. Thus we cannot have the case which results in $\reln{w_k^j}{iwo}{w_i^j}$.  

            \critic{red}{I do not think we can use WLOG for this and we must also write the case for writes below the read as they rely on property 8 for the proof.}

            Thus, for a given set of equal writes, once an implied write order is fixed, it remains fixed.
        \end{proof}

%------------------------------------------------------------------------------------------------------------------------------------------

        \begin{lemma}
            For a given set of equal writes whose implied write orders respect \ref{prop1},  new implied write order introduced among them by fixing other sets, if wrong, can be fixed and will remain fixed. 
            (A more formal statement required? Discuss with Viktor)        
        \end{lemma}

        \begin{proof}
            Suppose, for a given set of equal writes, say of the form $w^j$ all the implied write orders are fixed. Without loss of generality, let us consider them to be wrties above the read. We consider two threads $T_i$ and $T_k$, between which an implied write order is wrong. Let those writes be $w_i^l$ and $w_k^l$. 

            Once again, without loss of generality, let us consider the symmetric memory order between writes of $T_i$ and $T_k$ to be of the form $\reln{w_i}{smo}{w_k}$. Thus we have 
            \begin{align}
                \reln{w_i^j}{smo}{w_k^j} \ \wedge \ \reln{w_i^l}{smo}{w_k^l}
            \end{align}

            We assume write order is wrong between $w_i^l$ and $w_k^l$
            \begin{align}
                \reln{w_k^l}{iwo}{w_i^l}
            \end{align} 
            There cannot be an implied write order between $w_i^j$ and $w_k^j$, by \ref{prop5}. 

            \textbf{Case1: $w^l$ is below the read.}
            
                Part1: There exists an implied write order between another write $w^j$ and $w_i^j$. 

                    By \ref{prop7} and $\reln{w_k^l}{iwo}{w_i^l}$, thus satisfying $T_i$'s read, $w_i^j$ can only be involved in implied write orders of the form 
                    \begin{align}
                        \reln{w^j}{iwo}{w_i^j}
                    \end{align}
                    From our assumption of implied write orders among $w^j$ are fixed already, we have
                    \begin{align}
                        \reln{w^j}{smo}{w_i^j} 
                    \end{align} 
                    By \ref{SymMemO}, we then have $\reln{w^j}{smo}{w_k^j}$.

                    By \ref{prop3}, swapping $T_i$ and $T_k$ to fix $\reln{w_k^l}{iwo}{w_i^j}$, gives us the following new relations.
                    \begin{align}
                        \reln{w_i^l}{iwo}{w_k^j} \\
                        \reln{w^j}{iwo}{w_k^j}
                    \end{align}
                    Both these relations repsect our irreflexivity constraint \ref{prop1}. Thus, concluding this part. 
                    
                Part2: There exists an implied write order between another write $w^j$ and $w_k^j$. 
                    
                    By \ref{prop7 }$w_k^j$ can be involved in implied write orders of the form
                    \begin{align}
                        \reln{w_k^j}{iwo}{w^j} \\
                        \reln{w^j}{iwo}{w_k^j} 
                    \end{align}
                    The first is symmetric to Part1, hence we only consider the second form. 

                    By \ref{prop3}, on swapping $T_i$ and $T_k$, we get the following new relations:
                    \begin{align}
                        \reln{w_i^l}{iwo}{w_k^j} \\
                        \reln{w^j}{iwo}{w_i^j}
                    \end{align}

                    The second one may not be compliant to the irreflexivity condition \ref{prop1}, hence can be a new implied relation which is wrong. To show that this can be fixed and will remain fixed, we need to show that this is new and could have not occurred before while swapping threads to fix implied write orders of $w^j$. 
                    
                    Consider the original configuration we have 
                    \begin{align}
                        \reln{w_i^j}{smo}{w_k^j} \ \wedge \ \reln{w^j}{iwo}{w_k^j}
                    \end{align}
                    For $\reln{w^j}{iwo}{w_k^j}$ to have been there before, there must be a write $x$, say, such that 
                    \begin{align}
                        \reln{x}{iwo}{w_i^j} \ \wedge \ \reln{x}{iwo}{w^j}
                    \end{align}
                    By \ref{prop7}, such an $x$ cannot exist, Hence, such a relation could have not been there before, and hence was not fixed before.

                    Now that this new relation exists, fixing it, will keep it remain fixed, by our first part of proof (label them please. 
                    
            
            \textbf{Case2: $w^l$ is above the read.}
                
                Part1: There exists an implied write order between another write $w^j$ and $w_i^j$.

                Since $T_i$'s read is still free, $w_i^j$ can have implied write orders with $w^j$ of the form
                \begin{align}
                    \reln{w^j}{iwo}{w_i^j} \\
                    \reln{w_i^j}{iwo}{w^j} 
                \end{align}

                For the first form of relation, note that we also have by \ref{prop1} and by \ref{SymMemO}, 
                \begin{align}
                    \reln{w^j}{smo}{w_i^j} \ \wedge \ \reln{w^j}{smo}{w_k^j}
                \end{align}

                By \ref{prop3} on swapping two threads $T_i$ and $T_k$, we have 
                \begin{align}
                    \reln{w_i^l}{iwo}{w_k^j} \\
                    \reln{w^j}{iwo}{w_k^j}
                \end{align}

                Both relations respect our irreflexivity constraint.  Hence, maintaining that new implied write order relations with $w^j$ are not wrong.

                For the second form, while we swap threads $T_i$ and $T_k$, we get the following new relations (correct his)
                \begin{align}
                    \reln{w_i^l}{iwo}{w_k^j} \\
                    \reln{w_k^j}{iwo}{w^j}
                \end{align}
                The second relation may not be compliant to the irreflexivity condition, hence can be a new implied relation which is wrong. To show that this is new and could have not occurred before while swapping threads to fix implied write orders of $w^j$, consider the original configuration we have 
                \begin{align}
                    \reln{w_i^j}{smo}{w_k^j} \ \wedge \ \reln{w_i^j}{iwo}{w^j}
                \end{align}

                For $\reln{w^j}{iwo}{w_k^j}$ to have been there before, we need an event $x$ to connect, hence by possible implied write orders one can have with $w^j$ and $w_k^j$, we need an $x$ such that one of the conditions below hold
                \begin{align}
                    \reln{w^j}{iwo}{x} \ \wedge \ \reln{x}{iwo}{w_k^j} \\ 
                    \reln{x}{iwo}{w^j} \ \wedge \ \reln{x}{iwo}{w_k^j}
                \end{align}

                The first condition violates coherence ($po \cup rf_{-1}$ acylcic), while the second condition cannot hold as such an event $x$ cannot exist due to property x (number the properties).
                
                Thus, such a relation could have not been there before, and hence was not fixed. 

                \critic{red}{Write the above argument better please.}

                Part2: There exists an implied write order between another write $w^j$ and $w_k^j$.

                Since $T_k$'s read is already established in a reads-from relation, by \ref{prop7}, $w_k^j$ can have implied write orders with some $w^j$ only of the form 
                \begin{align}
                    \reln{w^j}{iwo}{w_k^j}
                \end{align}

                By \ref{prop3}, swapping thread identities $T_i$ and $T_k$ will give us the following relations 
                \begin{align}
                    \reln{w_i^l}{iwo}{w_k^l} \\ 
                    \reln{w^j}{iwo}{w_i^j} 
                \end{align}
                The second relation may not be compliant to the irreflexivity condition, hence can be a new implied relation which is wrong. To show that this is new and could have not occurred before while swapping threads to fix implied write orders of $w^j$, consider the original configuration we have 
                \begin{align}
                    \reln{w_i^j}{smo}{w_k^j} \ \wedge \ \reln{w^j}{iwo}{w_k^j}
                \end{align}

                For $\reln{w^j}{iwo}{w_i^j}$ to have been there before, we need an event $x$ to connect, hence by possible implied write orders one can have with $w^j$ and $w_k^j$. By \ref{prop7}, $w^j$ can only have relations of the form $\reln{w}{iwo}{w^j}$. Thusm we have the possible two forms of relations that could exist:
                \begin{align}
                    \reln{x}{iwo}{w^j} \ \wedge \ \reln{w_i^j}{iwo}{x} \\
                    \reln{x}{iwo}{w^j} \ \wedge \ \reln{x}{iwo}{w_i^j} \\ 
                \end{align}

                The second form cannot exist due to \ref{prop7}. While the first form could exist, it violates coherence ($po \cup rf_{-1}$ acylcic). 
                
                Thus, such a relation could have not been there before, and hence was not fixed. 

                \critic{red}{Although the soundness proof is written completely, it requires proper formatting and references to the properties and definitions we wrote above. This will make it concicse and a lot of symmetric arguments can be avoided.}
        \end{proof}

        

   

        \subsection{Implied Read Orders - A road to optimality?}

    %State examples showing that implied write orders though sorted can still have us explore more executions.
    %Take pics from notes.
    Let us start with the example below:
    \begin{figure}
        \centering
        \includegraphics{Equivalence_Example.pdf}
        \caption{Example program for read orders}
        \label{read_ord:ex1}
    \end{figure}

    Consider now two possible execution graphs of the above program as below:
    \begin{figure}
        \centering
        \includegraphics{Equivalence_Example.pdf}
        \caption{Two execution graphs of the program in Figure \ref{read_ord:ex1}}
        \label{read_ord:exec_ex1}
    \end{figure}

    Notice that both the above executions respect the irreflexivity constraint Prop \ref{prop1}.
    Naturally, we must then consider both executions while performing model checking.
    However, also notice that, on swapping threads $T_1$ and $T_3$ of the first execution graph, we do obtain the second execution graph.
    This, by our definition of symmetry would mean both the above executions are symmetric. 

    %Explain why the above symmetry exists.
    The only difference that is present in the two executions is from where the read values of $T_1$ and $T_3$ come from.
    In particular, note that $T_1$'s read value comes from the write of $T_4$, whereas that of $T_3$ comes from $T_2$.
    This by the standards of sequential consistency implies that $T_3$'s read occurred before that of $T_1$.
    While in that of the second execution, the order of reads is reverse. 
    We can then conclude that we could leverage the read orders to even further reduce the set of executions we need to verify.  

    %Using them, note the reader that implied read orders are incorrect in some and so sorting them might help.
    The above examples show us that read orders can also play a role in defining symmetric executions.
    The question however, is whether we can resolve all these read orders while also keeping the implied write orders to respect Prop \ref{prop1}.
    For this, we define a notion of a state, which represents the set of implied write orders and implied read orders that exist.    

    \paragraph{Resolving implied Read orders}

        It is important to note that we consider resolving implied read orders only when all the implied write orders are resolved.
        This means, if any implied write orders do not respect Prop \ref{prop1}, then we only concern ourselves to resolving them.
        Our definition of implied read orders is only to capture those patterns in executions where Prop \ref{prop1} is satisfied.
        
        %total order between reads of symmetric threads.
        \begin{definition}{Symmetric Read Order ($sro$)}
            \label{SymReadOrd}
            A strict partial order between reads of symmetric threads.
            
            Consider a set of symmetric threads $T_1 \equiv T_2 \equiv ... \equiv T_n$. 
            Each of these threads have exactly one read event, and multiple write events, all to the same memory, say $x$. 
            Then each read in the above threads are involved in a symmetric order, such that
            \begin{align*}
                \forall i \in [0, n-1] \ . \ \reln{r_i}{sro}{r_{i+1}}.
            \end{align*}

        \end{definition}

        %definition of implied read order. 
        \begin{definition}{Implied Read Order($iro$)}
            \label{ImpReadOrd}
            Binary relation between two reads in symmetric threads. 
            Its definition is a sequential composition as follows:
            \begin{align*}
                rf^{-1};smo;rf
            \end{align*} 

        \end{definition}

        \begin{definition}{Optimality Condition}
            \label{opt_cond}
            Using Def \ref{SymReadOrd}, our optimality requirement is that every execution graph has read orders that respect the symmetric read order.
            Formally, we can define it as the following \textit{irreflexivity constraint}:
            \begin{align*}
                sro;iro
            \end{align*}

        \end{definition}

        %We now define certain properties that would be useful for reasoning about read orders and their relation with implied write orders. 
%
        %%the three forms of implied read order using implied write order and symmetric memory order.
        %\begin{property}
        %    \label{inf-iwo}
        %    If an implied read order is derived using writes above the read, then xyz.
        %    If an implied read order is derived using writes below the read, then xyz.
        %\end{property}
%
        %\begin{proof}
        %    Xyz
        %\end{proof}
%
        %%every incorrect iro between two threads has no implied write orders between them 
        %\begin{property}
        %    \label{iro-no-iwo}
        %    If two threads have an "incorrect(one that does not respect our irreflexivity constraint)" implied read order between them, they do %not have any implied write order between them. 
        %    Because we are only concerned about swapping threads with incorrect implied read orders, we only consider them for any reversal of %fixed implied write order.     
        %\end{property}
%
        %\begin{proof}
        %    Xyz
        %\end{proof}
%
        %%implied read order is reversed when their corresponding thread identities are reversed
        %\begin{property}
        %    \label{iro-Rev}
        %    Implied read orders are reversed when swapping their respective thread identities.
        %\end{property}
%
        %\begin{proof}
        %    Xyz
        %\end{proof}
%
        %%PENDING REVIEW
        %\begin{property}
        %    \label{iro-partial-stability}
        %    Reversing an implied read order, does not change implied read order of threads whose writes have been read by these respective reads.%(formally specify it)
        %\end{property}
%
        %\begin{proof}
        %    Xyz
        %\end{proof}
%
        %%acylic is needed to show that they can be resolved respecting our irreflexivity constraint.
        %\begin{property}
        %    \label{acyclic-iro}
        %    Implied read orders are acyclic.
        %\end{property}
%
        %\begin{proof}
        %    Xyz
        %\end{proof}

        %Mention the strategy you would adopt here. 
        The question now is whether we can resolve all such implied read orders to respect Def \ref{opt_cond}. 
        Naturally, the strategy we can adopt could be same as that for soundness: To check whether resolved implied read orders remain stable while resolving others. 
        Unfortunately, this does not turn out to be true.
    

    \paragraph{Forward Progress of Resolving Read Orders}

        We cannot approach optimality the same way as we did Soundness.
        This is because stability of implied read orders is not present while resolving them.
        The following program's execution is one such example.
        
        %Diagram from Ipad with program execution stages, showing clearly that read orders incorrect are reintroduced. 
        \critic{red}{The board example (ipad) is a perfect one to show this fact.}
        \begin{figure}
            \centering
            \includegraphics{Equivalence_Example.pdf}
            \caption{Example of program execution where resolving read orders does not keep read orders stable.}
            \label{read_ord:instability}
        \end{figure}

        In the above figure, notice that the read orders between $T_a$ and $T_b$ become incorrect once again while we resolve other implied read orders.
        
        \critic{red}{Perhaps also consider showing another example just to make them more assured of the fact that even prescribing an order in which we resolve these read orders, it does not matter.}

        \critic{red}{Show example where after fixing a read order and another, the previous one is present again.}

        Instead, what we will try to prove is \textbf{Forward Progress}; does resolving read orders eventually result in the irreflexivity condition to hold? 
        For this, we split our task into two parts, which are stated as two lemmas below. 
        Before that, we will introduce the notion of state here, to more formally state our requirement. 
        
        %Define the state here as a set of implied write orders and implied read orders.
        A state $S$ is a set of implied write orders and implied read orders.
        A transition from such a state to another involves swapping two thread identities, which may result in removal and addition of new implied write/read orders.
        For instance, transitioning from state $S$ to $S'$ after swapping threads $A$ and $B$ would be given as 
        \begin{align*}
            \reln{S}{(A,B)}{S'}.
        \end{align*}
        To depict multiple transitions of the above we will represent as $\reln{S}{*}{S'}$.
        \textit{Sound(S)}  depicts that the state $S$ respects Prop \ref{prop1}.
        
        
        \critic{red}{Show lemmas first, then the theorem. Prove the theorem first, assuming the lemmas hold. As the argument is quite intuitive. Then prove lemma 2 and show that it does not hold due to counter example. Then conclude that we cannot achieve optimality by resolving implied read orders. This would require changing the structure of the following content slightly.}

        Our first part is on the assumption that while resolving implied read orders, we do not encounter any incorrect implied write order. 
        %lemma to show that if an incorrect read order previously resolved is introduced by resolving another read order, then it implies the the currently resolved one remains unresolved before and after resolving the previous one
        \begin{lemma}
            \label{iro-stability}
            Consider a state $S$ such that $Sound(S) \ \wedge \ \reln{r_j}{iro}{r_i} \ \wedge \ i < j$.
            Consider state $S'$ after swapping threads $T_i$ and $T_j$ and assume that no incorrect implied write orders are introduced in doing so. 
            Consider also that $\reln{r_l}{iro}{r_k} \in S' \ \wedge \ k < l$.
            If we swap threads $T_l$ and $T_k$ giving resultant state $S''$, then 
            \begin{align*}
                \reln{r_j}{iro}{r_i} \in S'' 
                \Rightarrow
                \reln{r_l}{iro}{r_k} \in S
            \end{align*} 
        \end{lemma}

        \begin{proof}

            \critic{red}{Write this proof first. And while proving, show that one case is a counter example.}

            \begin{itemize}
                \item Case wise analysis.
                \item Infer different ordering relations to show that said implied read order is stable.
            \end{itemize}

        \end{proof}


        Our second lemma relaxes the assumption of the above lemma, thus allowing implied write orders to be incorrect. 
        This would mean, we will have to resolve them first, before once again resolving implied read orders.
        \begin{lemma}
            \label{iwo-unres-iro}
            Consider state $S$ such that $Sound(S) \ \wedge \ \reln{r_j}{iro}{r_i} \ \wedge \ i < j$.
            Consider state $S'$ that results after swapping threads $T_i$ and $T_j$. 
            Then if $\neg Sound(S')$, the new implied write orders introduced were not resolved in any previous state $S''$ such that $\reln{S''}{*}{S}$. 
        
            \critic{red}{The word resolved is important as it means there wasn't any step in state transition that involved swapping the two threads to resolve implied write order between them.}
        \end{lemma}


        \begin{proof}
            \begin{itemize}
                \item Case wise analysis
                \item Each case is proof by contradiction
                \item Main argument is the fixed order in which we fix the implied orders (writes first then reads)
            \end{itemize}    
        \end{proof}

        If the above lemmas hold, then we can resolve all such read orders to respect Def \ref{opt_cond}.
        The following theorem states this:
        %STOP HERE. 
        \begin{theorem}
            \label{fwdprog-iro}
            We can resolve all implied read orders to respect the irreflexivity constraints imposed on them.
        \end{theorem}

        \begin{proof}

            We begin the proof by considering a state $S$ representing the set of implied write/read orders we have in the current execution. 
            We further assume that all implied write orders have been resolved in $S$ ($Sound(S)$). 
            Now, we divide our analysis into two cases, which may take place while resolving read orders.
            For each case, we prove Forward Progress by that of contradiction.
            We try to see if we can reach the same state $S$ after resolving some orders, thus implying an endless cycle. 
            \begin{itemize}
                \item Case 1: If any incorrect implied write orders introduced in the process of resolving read orders. 

                    By Theorem \ref{iwo-sound}, we can resolve them. 
                    By Lemma \ref{iwo-unres-iro}, every implied write order introduced will be new.
                    Thus, by contradiction, we cannot retain the original state $S$ from which we resolve the same set of implied read orders. 

                \item Case 2: If no incorrect implied write orders introduced in the process of resolving read orders.

                    To retain the original state with a particular unresolved implied read order, we can only swap threads with incorrect implied read orders as per our assumption. 
                    By Lemma \ref{iro-stability}, obtaining a previously resolved iro as unresolved will in the process resolve another iro which remained stable while resolving the previous one.
                    By contradiction, this is not the same state as $S$. 
            \end{itemize}

            By Case 1 and 2, one cannot reach the same initial state.
            Because implied write/read orders are finite, the number of states are finite, and since we can only reach a state exactly once, 
            we can resolve all implied read orders in the process. 

            Hence proved.

        \end{proof}

        Surprisingly, Lemma \ref{iro-stability} does not hold. 
        The following example of an execution shows that we are stuck in a cycle resolving implied read orders.


        \begin{figure}
            \centering
            \includegraphics{Equivalence_Example.pdf}
            \caption{Set of executions, showing transitions while resolving read orders, which results in a cycle.}
            \label{iro:counter_example}
        \end{figure}

        In particular, from the above example, we note that:
        \begin{itemize}
            \item We cannot resolve all the implied read orders to respect our irreflexivity constraint for optimality.
            \item This is due to a cycle that is generated while resolving them using Prop \ref{iro-Rev}.
        \end{itemize}

        Thus, we cannot achieve optimality by resolving read orders. 

     


\end{document}
    



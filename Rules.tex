\subsection{From Examples to Precise Rules}

    \paragraph{Notations}
        \begin{itemize}
            \item $T_i$ denotes thread number $i$.
            \item $T_i \equiv T_j$ means both threads have same code.
            \item $w_i^j$ is the $j^{th}$ event in thread $i$ which is a write.
            \item $r_i^j$ is the $j^{th}$ event in thread $i$ which is a read. 
        \end{itemize}

    
    \paragraph{A few definitions for our use}

    \begin{definition}{Program Order(\emph{po})}
        \label{ProgO}
        Total order between events in the same thread. Respects the execution order between events in the same thread. 
    \end{definition}

    \begin{definition}{Symmetric Memory Order (\emph{smo})}
        \label{SymMemO}
        A strict partial order between writes in a set of symmetric threads. 
        
        Consider a set of symmetric threads $T_1 \equiv T_2 \equiv ... \equiv T_n$. 
        Each of these threads have exactly one read event, and multiple write events, all to the same memory, say $x$. 
        Then each write in the above threads are involved in a symmetric order, such that. 
        \begin{align*}
            \forall i \in [0, n-1] \ . \ \reln{w_i^j}{smo}{w_{i+1}^j}
        \end{align*}
        Where j denotes the $j^{th}$ event in any of the threads, which is a write.
    \end{definition}

    \critic{blue}{Perhaps should put examples for the above defintion.}

    \begin{definition}{Reads-From (\emph{rf})}
        \label{ReadF}
        Binary relation that links a read to a write from which its value comes. Note that for our purpose, this relation is functional.
        For example, if a read $r_i^j$ gets its read value from write $w_k^l$, then we have the relation. 
        \begin{align*}
            \reln{w_k^l}{rf}{r_i^j}
        \end{align*}
    \end{definition}

    \paragraph{Main Rule}
        Using the above setup, our intention is to explore lesser execution graphs leveraging the symmetry that can result due to swapping of thread identities. For this, we enforce a restriction on possible $\stck{_{rf}}$ relations that are to be considered \emph{valid}. A valid $\stck{_{rf}}$ relation is one that respects the following \textbf{irreflexivity constraints}. 
        \begin{align*}
            smo;rf;po \\
            smo;po;rf^{-1} 
        \end{align*}

        \critic{blue}{Perhaps consider labelling this rule}

        \critic{blue}{We can add here more as we go about to prove completeness.}

        \critic{red}{Recall examples to show how our analysis through examples satisfy the above irreflexivity constraint.}

        

